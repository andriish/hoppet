% ======================================================================
\section{Evolution including QED contributions}
\label{sec:qed-evolution}

The combined QED+QCD evolution, as implemented in \hoppet since
version 1.3.0 (and earlier in a dedicated \ttt{qed} branch), was first
described in
Refs.~\cite{Manohar:2016nzj,Manohar:2017eqh,Buonocore:2020nai,Buonocore:2021bsf}.
%
The determination of which contributions to include follows a
consistent approach based on the so-called ``phenomenological'' counting
scheme.
%
Within this scheme, one considers the QED coupling $\alpha$ to be of order $\as^2$, and takes the
photon (lepton) PDF to be of order $\alpha L$ ($\alpha^2 L^2$), where
$L$ is the logarithm of a ratio of the factorisation scale to a typical hadronic scale and is considered to be of order $L \sim 1/\as$. In contrast, quark and gluon
PDFs are considered to be of order $(\as L)^n={\cal O}(1)$.\footnote{The
  above counting is to be contrasted with a ``democratic'' scheme, in
  which one considers $\alpha \sim \as \sim 1/L$ and where the aim would be to maintain
  the same loop order across all couplings and splitting functions,
  regardless of the relative numbers of QCD and QED couplings that
  they involved.}
%
From this point of view, NNLO (3-loop) QCD evolution provides control of
terms of order up to $\as^{n+2} L^n \sim \as^2$.
%
To achieve a corresponding accuracy when including QED contributions,
\hoppet has been extended to account for
\begin{enumerate}
\item \label{item:qed1} 1-loop QED splitting functions~\cite{Roth:2004ti}, which first
  contribute at order $\alpha L \sim \as$, i.e.\ count as NLO QCD
  corrections;
  
\item \label{item:qed2} 1-loop QED running coupling, including lepton and quark
  thresholds, which first contributes at order $\alpha^2 L^2 \sim
  \as^2$, i.e. like NNLO QCD; 
  
\item \label{item:qed3} 2-loop mixed QCD-QED splitting
  functions~\cite{deFlorian:2015ujt},
  %\footnote{One-loop triple collinear splitting functions with photons have been computed in refs~\cite{Sborlini:2014mpa,Sborlini:2014kla}};,
  which first contribute at order
  $\alpha \as L \sim \as^2$, i.e.\ count as NNLO QCD corrections;

\item \label{item:qed4} optionally, the 2-loop pure QED $P_{\ell q}$ splitting
  function~\cite{deFlorian:2016gvk}, which brings absolute accuracy
  $\alpha^2 L\sim \as^3$ to the lepton distribution (which starts at
  $\alpha^2 L^2 \sim \as^2$).\footnote{The implementation of the 2-loop $P_{\ell q}$ splitting
  function in the hoppet code (\ttt{Plq\_02} in the code) was carried out by Luca Buonocore.}
  %
  In an absolute counting of accuracy, this is not needed.
  % 
  However, if one wants lepton distributions to have the same relative
  NLO accuracy as the photon distribution, it should be
  included.
\end{enumerate}
%
The code could be extended systematically to aim at a higher
accuracy. For instance, if one wished to reach N$^3$LO accuracy in the
phenomenological counting, one would need to include 3-loop mixed
QCD-QED splitting functions at order $\alpha \as^2$ which contribute
at order $\alpha \as^2 L \sim \as^3$, but are currently not
available, and the well-known 2-loop QED running coupling at order $\alpha \alpha_s$ which contributes at order $\alpha^2 \as L^2 \sim \as^3$.
%
%\comment{Discuss what is needed to go effectively to N3LO? $\alpha^2$
%  splitting exists, $\alpha\as^2$ does not?}
%


\subsection{Implementation of the QED extension}

\myparagraph{QED coupling}

A first ingredient is the setup of the QED coupling object, defined in
\ttt{module qed\_coupling}:
\begin{lstlisting}
  type qed_coupling 
     real(dp) :: m_light_quarks
     real(dp) :: mc, mb, mt
     integer  :: n_thresholds
     integer  :: nflav(3, 0:n_thresholds) ! first index: 1 = nleptons, 2=ndown, 3=nup
     real(dp) :: thresholds(0:n_thresholds+1)
     real(dp) :: b0_values(n_thresholds)
     real(dp) :: alpha_values(n_thresholds)
  end type qed_coupling
\end{lstlisting}
%
This is initialized through a call to 
\begin{lstlisting}
  subroutine InitQEDCoupling(coupling, m_light_quarks, &
   &                           m_heavy_quarks(4:6) [,value_at_scale_0])
    type(qed_coupling), intent(out) :: coupling
    real(dp),           intent(in)  :: m_light_quarks, m_heavy_quarks(4:6)
    real(dp), optional, intent(in)  :: value_at_scale_0 ! defaults to alpha_qed_scale_0
    [...]
  end subroutine InitQEDCoupling
\end{lstlisting}

It initialises the parameters relevant to the QED coupling and its
running.  The electromagnetic coupling at scale zero is set by default
to its PDG~\cite{ParticleDataGroup:2022pth} value, unless the optional
argument \ttt{value\_at\_scale\_0} is provided, in which case the
latter is taken as the value of the QED coupling at zero momentum.

The running is performed at leading order level using seven
thresholds: a common effective mass for the three light quarks
(\ttt{m\_light\_quarks}), the three lepton masses (hard-coded to their
PDG values~\cite{ParticleDataGroup:2022pth} in the
\ttt{src/qed\_coupling.f90} file), and the three masses of the heavy
quarks (\ttt{m\_heavy\_quarks(4:6)}).
%
The common value of the light quark masses is used to mimic the
physical evolution in the region $0.1\GeV \lesssim \mu \lesssim
1\GeV$, which involves hadronic states.
%
Using a value of $0.109\GeV$ generates QED coupling values at
the masses of the $\tau$ lepton ($1/133.458$)and $Z$-boson ($1/127.952$) that
agree with the PDG $\MSbar$  ones ($1/(133.471\pm0.007)$ and $1/(127.951\pm0.009)$)
with a relative $\sim 10^{-4}$ accuracy. \gps{We don't really have a
  QED renormalisation scheme defined and $1/137$ that we use for $\alpha(0)$
  is not, I think, an $\MSbar$ value; hence the rephrasing putting
  MSbar in front of PDG rather than in front of our coupling.}
%
%Lepton masses are hard-coded to their PDG
%values~\cite{ParticleDataGroup:2022pth} in the
%\ttt{src/qed\_coupling.f90} file.
%
%The heavy-quark masses \ttt{quark\_masses(4:6)} are instead required as
%non-optional arguments.

The quark and lepton masses are used to set all thresholds where the
fermion content changes. This information is then used to set
the array \ttt{nflav(3,0:n\_thresholds)}, where \ttt{nflav(1:3,i)}
contains an integer array with the number of leptons, down, and up
quarks at a given $Q^2$ such that \ttt{threshold(i-1)}$<Q^2<$
\ttt{threshold(i)}.
%
The \ttt{threshold(1:7)} entries are active thresholds, while
\ttt{threshold(0)} is set to zero and \ttt{threshold(8)} to an
arbitrary large number (currently $10^{200}$).
%
The integer array
\ttt{nflav(1:3,0:n\_thresholds)} is then used to compute the
$\beta_{0,\rm QED}$ function \ttt{b0\_values(1:n\_thresholds)} at the
seven threshold values and this is finally used to compute the value
of the QED coupling \ttt{alpha\_values(1:n\_thresholds)} at the
threshold values.
%
The function \ttt{Delete(qed\_coupling)} is also provided for
consistency, although in this case it does nothing.
%
After this initialization, the function \ttt{Value(qed\_coupling,mu)} returns the QED coupling at
scale $\mu$.

\myparagraph{QED splitting matrices}

The QED splitting matrices are stored in the object

\begin{lstlisting}
  type qed_split_mat
     type(qed_split_mat_lo)   :: lo
     type(qed_split_mat_nlo)  :: nlo
     type(qed_split_mat_nnlo) :: nnlo
  end type qed_split_mat
\end{lstlisting}
defined in \ttt{qed\_objects.f90}. 
This contains the LO, NLO and NNLO splitting matrices 
\begin{lstlisting}
  ! a leading-order splitting matrix (multiplies alpha/2pi)
  type qed_split_mat_lo
     type(grid_conv) :: Pqq_01, Pqy_01, Pyq_01, Pyy_01
     integer         :: nu, nd, nl, nf
  end type qed_split_mat_lo

  ! a NLO splitting matrix (multiplies (alpha alpha_s)/(2pi)^2)
  type qed_split_mat_nlo
     type(grid_conv) :: Pqy_11, Pyy_11, Pgy_11
     type(grid_conv) :: Pqg_11, Pyg_11, Pgg_11
     type(grid_conv) :: PqqV_11, PqqbarV_11, Pgq_11, Pyq_11
     integer         :: nu, nd, nl, nf
  end type qed_split_mat_nlo

  ! a NNLO splitting matrix (multiplies (alpha/(2pi) )^2) 
  ! contains only Plq splitting!
  type qed_split_mat_nnlo 
     type(grid_conv) :: Plq_02
     integer         :: nu, nd, nl, nf
  end type qed_split_mat_nnlo
\end{lstlisting}     

Above, \ttt{y} denotes a photon and the pairs of integers \ttt{01},
\ttt{11} and \ttt{02} denote the orders in the QCD and QED couplings,
respectively. Besides the number of quarks \ttt{nf}, these splitting
matrices also need the number of up-type (\ttt{nu}) and down-type
quarks (\ttt{nd}) separately, and the number of leptons (\ttt{nl}).
%
Note that the splitting functions of order $\alpha$ (i.e.\ \ttt{01})
for the leptons are simply obtained from the ones
involving quarks by adjusting colour factors and couplings.

A call to the subroutine 
\begin{lstlisting}     
  subroutine InitQEDSplitMat(grid, qed_split)
    use qed_splitting_functions
    type(grid_def),      intent(in)    :: grid
    type(qed_split_mat), intent(inout) :: qed_split
    [...]
  end subroutine InitQEDSplitMat
\end{lstlisting}     
initializes the \ttt{qed\_split\_mat} object \ttt{qed\_split} and sets all QED
splitting functions on the given \ttt{grid}.
%
The above QED objects can be used for any sensible value of the
numbers of flavours, on the condition that one first registers
the current number of flavours with a call to
\begin{lstlisting}
  QEDSplitMatSetNf(qed_split, nl, nd, nu)
\end{lstlisting}
where \ttt{nl}, \ttt{nd} and \ttt{nu} are respectively the current
numbers of light leptons, down-type and up-type quarks.
%
In practice, this is always handled internally by the QED-QCD
evolution routines, based on the thresholds encoded in the QED
coupling.\footnote{While the QCD splitting functions are initialised
  and stored separately for each relevant value of $n_f$, in the QED
  case the parts that depend on the numbers of flavours are separated
  out.
  %
  Only when the convolutions with PDFs are performed are the relevant
  $n_f$ and electric charge factors included.}
%
The one situation where a user would need to call this routine
directly is if they wish to manually carry out convolutions of the
QED splitting functions with a PDF.

Subroutines \ttt{Copy} and \ttt{Delete} are also provided for the
\ttt{qed\_split\_mat} type. As in the pure QCD case, convolutions with
QED splitting functions can be represented by the \ttt{.conv.} operator or
using the product sign \ttt{*}.


\myparagraph{PDF arrays with photons and leptons}

A call to the subroutine
\begin{lstlisting}  
  subroutine AllocPDFWithPhoton(grid, pdf)
    type(grid_def), intent(in) :: grid
    real(dp),       pointer    :: pdf(:,:)
    [...]
  end subroutine AllocPDFWithPhoton
\end{lstlisting}
allocates PDFs (\ttt{pdf}) including photons, while a call to
\begin{lstlisting}  
  subroutine AllocPDFWithLeptons(grid, pdf)
    type(grid_def), intent(in) :: grid
    real(dp),       pointer    :: pdf(:,:)
    [...]
    end subroutine AllocPDFWithLeptons
\end{lstlisting}
allocates PDFs including both photons and leptons.
%
The two dimensions of the \ttt{pdf} refer respectively to
the index of the $x$ value in the \ttt{grid}, and to the flavour index.
The flavour indices for photons and leptons %in the arrays \ttt{pdf}
are
given by
\begin{lstlisting}  
  integer, parameter, public :: iflv_photon   =  8
  integer, parameter, public :: iflv_electron =  9
  integer, parameter, public :: iflv_muon     = 10
  integer, parameter, public :: iflv_tau      = 11
\end{lstlisting}
where each \ttt{pdf(:,9:11)} contains the sum of a lepton and
anti-lepton flavour (which are identical). Note that if one were to
extend the calculation of lepton PDFs to higher order in $\alpha$,
then an asymmetry in the lepton and anti-lepton distribution would
arise, due to the \ttt{Plq\_03} splitting function\ak{I think this
  needs more explanation}.
%
\gps{Giulia had found a ref that gave a nice explanation, maybe we can
  add that ref here?}
%
In that case it would become useful
to have separare indices for leptons and anti-leptons.

% \footnote{
%   Were one to use \hoppet to obtain the distribution of
%   partons inside a lepton, then it would become useful to separate
%   lepton and anti-lepton PDFs.
%   % 
%   However further considerations arise, notably for the treatment of
%   the $1-x \ll 1$ region.\commentgz{explain better?} 
%   % 
%   See for example discussion in Ref.~\cite{Frixione:2023gmf}\commentgz{update refs}.}

The subroutine \ttt{AllocPDFWithPhotons} allocates the \ttt{pdf}
array with the flavour index from -6 to 8,
while in the subroutine \ttt{AllocPDFWithLeptons},
the flavour index extends from -6 to 11.


\myparagraph{PDF tables with photons and leptons}

Next one needs to prepare a \ttt{pdf\_table} object forming the
interpolating grid for the evolved PDF's. We recall that the
\ttt{pdf\_table} object contains an underlying array \ttt{pdf\_table\%tab(:,:,:)},
where the first index loops over $x$ values, the second loops over
flavours and the last loops over $Q^2$ values.


This is initialized by a call to 
\begin{lstlisting}
  subroutine AllocPdfTableWithLeptons(grid, pdftable, Qmin, Qmax & 
  & [, dlnlnQ] [, lnlnQ_order] [, freeze_at_Qmin])
   use qed_objects
   type(grid_def),    intent(in)    :: grid
   type(pdf_table),   intent(inout) :: pdftable
   real(dp), intent(in)             :: Qmin, Qmax
   real(dp), intent(in), optional   :: dlnlnQ
   integer,  intent(in), optional   :: lnlnQ_order
   logical,  intent(in), optional   :: freeze_at_Qmin
   [...]
   end subroutine AllocPdfTableWithLeptons
\end{lstlisting}
that is identical to the one without photon or leptons, the only difference is that the
maximum pdf flavour index in \ttt{pdf\_table\%tab} now includes the photon and leptons.
%
An analogous subroutine \ttt{AllocPdfTableWithPhoton} includes the
photon but no leptons.

\myparagraph{Evolution with photons and leptons}


To fill a table via an evolution from an initial scale, one calls the subroutine 
\begin{lstlisting}
    subroutine EvolvePdfTableQED(table, Q0, pdf0, dh, qed_split, &
    &  coupling, coupling_qed, nloop_qcd, nloopqcd_qed, with_Plq_nnloqed)
       type(pdf_table),        intent(inout) :: table
       real(dp),               intent(in)    :: Q0
       real(dp),               intent(in)    :: pdf0(:,:)
       type(dglap_holder),     intent(in)    :: dh
       type(qed_split_mat),    intent(in)    :: qed_split
       type(running_coupling), intent(in)    :: coupling
       type(qed_coupling),     intent(in)    :: coupling_qed
       integer,                intent(in)    :: nloop_qcd, nloopqcd_qed
       logical,  optional,     intent(in)    :: with_Plq_nnloqed
       [...]
    end subroutine EvolvePdfTableQED   
\end{lstlisting}
where \ttt{table} is the output, \ttt{Q0} the initial scale and
\ttt{pdf0} is the PDF at the initial scale.
%
We recall that the lower and upper limits on 
scales in the table are as set at initialisation time for the table.
%
When \ttt{nloopqcd\_qed} is set to 1 (0) mixed QCD-QED effects are
(are not) included in the evolution.
%
Setting the variable \ttt{with\_Plq\_nnloqed=.true.}\ includes also the NNLO $P_{lq}$
splittings in the evolution.

To perform the evolution \ttt{EvolvePdfTableQED} calls the routine 
\begin{lstlisting}
    subroutine QEDQCDEvolvePDF(dh, qed_sm, pdf, coupling_qcd, coupling_qed,&
    &                     Q_init, Q_end, nloop_qcd, nqcdloop_qed, with_Plq_nnloqed)
       type(dglap_holder),     intent(in), target :: dh
       type(qed_split_mat),    intent(in), target :: qed_sm
       type(running_coupling), intent(in), target :: coupling_qcd
       type(qed_coupling),     intent(in), target :: coupling_qed
       real(dp),               intent(inout)      :: pdf(0:,ncompmin:)
       real(dp),               intent(in)         :: Q_init, Q_end
       integer,                intent(in)         :: nloop_qcd
       integer,                intent(in)         :: nqcdloop_qed
       logical,  optional,     intent(in)         :: with_Plq_nnloqed
       [...]
       end subroutine QEDQCDEvolvePDF
\end{lstlisting}
Given the \ttt{pdf} at an initial scale \ttt{Q\_init}, it
evolves it to scale \ttt{Q\_end}, overwriting the \ttt{pdf} array. 
%
In order to get interpolated PDF values from the table we use the
\ttt{EvalPdfTable\_*} calls, described in
\ifreleasenote
Section~7.2 of Ref.~\cite{Salam:2008qg}.
\else
Section~\ref{sec:acc_table}.
\fi
%
However the \texttt{pdf} array that is passed as an argument and that
is set by those subroutines should range not from \texttt{(-6:6)} but
instead from \texttt{(-6:8)} if the PDF just has photons and
\texttt{(-6:11)} if the PDF also includes leptons.\footnote{Index $7$
is a historical artefact associated with internal \hoppet bookkeeping
and should be ignored in the resulting \ttt{pdf} array.}

Note that at the moment, when QED effects are included, cached
evolution is not supported.


\subsection{Streamlined interface with QED effects}
The streamlined interface including QED effects works as in the case of pure QCD evolution.
One has to add the following call
\begin{lstlisting}
  logical use_qed, use_qcd_qed, use_Plq_nnlo
  ...
  call hoppetSetQED(use_qed, use_qcd_qed, use_Plq_nnlo)
\end{lstlisting}
before using the streamlined interface routines.
%
The \ttt{use\_qed} argument turns QED evolution on/off at order
${\cal O}(\alpha)$ (i.e.\ items \ref{item:qed1} and \ref{item:qed2} in
the enumerated list at the beginning of
Sec.~\ref{sec:qed-evolution}).
%
The \ttt{use\_qcd\_qed} one turns
mixed QCD$\times$QED effects on/off in the evolution (i.e.\ item \ref{item:qed3}) and \ttt{use\_Plq\_nnlo} turns 
the order $\alpha^2 P_{\ell q}$ splitting function on/off (i.e.\
item \ref{item:qed4}).
%
Without this call, all QED corrections
are off.

With the above, the streamlined interface can then be used as normal.
%
E.g.\ by calling the \ttt{hoppetEvolve(...)} function to fill the PDF
table and \ttt{hoppetEval(x,Q,f)} to evaluate the PDF at a given $x$
and $Q$.
%
Note that the \ttt{f} array in the latter call must be
suitably large, e.g.\ \ttt{f(-6:11)} for a PDF with leptons.
%
An example of the streamlined interface being used with QED evolution
can be found in
\begin{quote}
  \repolink{example_f90/tabulation_example_qed_streamlined.f90}{example\_f90/tabulation\_example\_qed\_streamlined.f90}\\
  \repolink{example_f77/cpp_tabulation_example_qed.cc}{example\_f77/cpp\_tabulation\_example\_qed.cc}
\end{quote}
\gps{added this last paragraph}
\gps{NB: to make links work, I've hard-coded them to the current
  branch, but we should switch over to the master/main branch}

%======================================================================
%======================================================================
\section{Hadronic Structure Functions}
\label{sec:structure-funcs}
As of \hoppet version 1.3.0 the code provides access to the hadronic
structure functions. The structure functions are expressed as
convolutions of a set of massless hard coefficient functions and PDFs,
and make use of the tabulated PDFs and streamlined interface as
described in the sections above.\commentgz{there are no sections above here}
The structure functions are provided
such that they can be used directly for cross section computations in
DIS or VBF, the latter of which has already been implemented in the
{\tt proVBFH}
package~\cite{Cacciari:2015jma,Dreyer:2016oyx,Dreyer:2018qbw,Dreyer:2018rfu},
for which this work was originally developed. \gps{``work developed''
  is weird phrasing}

The structure functions have been found to be in good agreement with
those that can be obtained with APFEL++~\cite{Bertone:2017gds} (at
the level of $10^{-4}$ relative precision).
%
The benchmarks with APFEL++ and the code used to carry
them out are described in detail in Ref.~\cite{bertonekarlberg} and
can be found in
\masterlink{benchmarking/structure\_functions\_benchmark\_checks.f90}.
%
Technical details on the implementation of the structure functions in
\hoppet can also be found therein, and in
Refs.~\cite{Dreyer:2016vbc,Karlberg:2016zik}. A small sample program
that prints out the structure functions can be found in
\masterlink{example\_f90/structure\_functions\_example.f90}.

The structure functions have been implemented including only QCD corrections up to
N$^3$LO\footnote{With the caveat that the 4-loop splitting functions
which are needed to claim this accuracy are not currently fully
known. At the time of writing partial results have been presented in
Refs.~\cite{Moch:2021qrk,Falcioni:2023luc,Falcioni:2023vqq,Gehrmann:2023cqm,Falcioni:2023tzp,Moch:2023tdj,Gehrmann:2023iah}. The
splitting functions up to three loops were already implemented in
\hoppet as of version 1.1.0, and can be found
in~\cite{Furmanski:1980cm,Curci:1980uw,NNLO-NS,NNLO-singlet}
%
\gps{Last sentence is already in the main text}
} using
both the exact and parametrised coefficient functions found in
Refs.~\cite{vanNeerven:1999ca,vanNeerven:2000uj,Moch:2004xu,Vermaseren:2005qc,Moch:2008fj,Davies:2016ruz,Blumlein:2022gpp},
and the splitting functions up to three-loops already implemented in
\hoppet as of version
1.1.0~\cite{Furmanski:1980cm,Curci:1980uw,NNLO-NS,NNLO-singlet}. Additionally
the running coupling in \hoppet has been extended to four loops using
the results of
Ref.~\cite{vanRitbergen:1997va,Czakon:2004bu}.\footnote{The
implementation does not allow for non-standard values of the QCD
Casimir numbers \gps{numbers $\to$ invariants?} beyond three loops.}

\subsection{Initialisation}
\label{sec:structure-funcs-init}

The structure functions can be accessed from inside the module
\ttt{structure\_functions}. They can also be accessed through the
streamlined interface by prefixing \ttt{hoppet} as described in
section~\ref{sec:structure-functions-streamlined}.
%
The description below corresponds to a high-level interface, which
relies on elements such as the \ttt{grid} and splitting functions
having been initialised in the streamlined interface, through a call
to \ttt{hoppetStart} or \ttt{hoppetStartExtended}, cf.\
\ifreleasenote
Section~8 of Ref.~\cite{Salam:2008qg}.\footnote{Users needing a lower level
  interface should inspect the code in \masterlink{src/structure\_functions.f90}.}
\else
Section~\ref{sec:vanilla}.\footnote{Users needing a lower level
  interface should inspect the code in \masterlink{src/structure\_functions.f90}.}
\fi
%
After this initialisation has been carried out one calls
\begin{lstlisting}
  call StartStrFct(order_max [, nflav] [, xR] [, xF] [, scale_choice] &
                  & [, constant_mu] [ ,param_coefs] [ ,wmass] [ ,zmass])
\end{lstlisting}
specifying as a minimum the perturbative order --- currently
\ttt{order\_max} $ \le 4$ (\ttt{order\_max} $ =1$ corresponds to LO).

If \ttt{nflav} is not passed as an argument, the structure functions
are initialised to support a variable flavour-number scheme (the
masses that are used at any given stage will be those set in the
streamlined interface).
%
Otherwise a fixed number of light flavours is used, as indicated by
\ttt{nflav}, which speeds up initialisation.
%
Note that specifying a variable flavour-number scheme only has an
impact on the evolution and on $n_f$ terms in the coefficient
functions.
%
The latter, however always assume massless quarks.
%
Hence in a variable flavour-number scheme the structure functions
should only be considered reliable for $Q \gg m$.

Together \ttt{xR}, \ttt{xF}, \ttt{scale\_choice}, and \ttt{constant\_mu} control
the renormalisation and factorisation scales and the degree of
flexibility that will be available in choosing them at later stages.
%
Specifically (integer) \ttt{scale\_choice} argument can take several values:
\begin{itemize}
\item \ttt{scale\_choice\_Q} (default) means that the code will always
  use $Q$ multiplied by \ttt{xR} or \ttt{xF} as the renormalisation
  and factorisation scale respectively (with \ttt{xR} or \ttt{xF} as
  set at initialisation).
\item \ttt{scale\_choice\_fixed} corresponds to a fixed scale
  \ttt{constant\_mu}, multiplied by \ttt{xR} or \ttt{xF} as set at
  initialisation.
\item \ttt{scale\_choice\_arbitrary} allows the user to choose
  arbitrary scales at the moment of evaluating the structure
  functions.
  %
  In this last case, the structure functions are saved as arrays
  separated by perturbative order and with dedicated additional arrays
  for terms proportional to logarithms of $Q/\mu_F$. 
  % 
  This makes for a slower evaluation compared to the two other
  scale choices.
\end{itemize}
%
% The default value for \ttt{scale\_choice} is \ttt{scale\_choice\_Q} which means that $Q$
% multiplied by \ttt{xR} or \ttt{xF} is used as the renormalisation or
% factorisation scale respectively.
% %
% The choice \ttt{scale\_choice\_fixed} corresponds to a fixed scale
% \ttt{constant\_mu} (multiplied by \ttt{xR} or \ttt{xF}).
% %
% Should a user wish to use some arbitrary scale choice,
% \ttt{scale\_choice} should be set to \ttt{scale\_choice\_arbitrary}.
% %
% \gps{Do we want to introduce some constants such as
%   \ttt{scale\_choice\_Q=1}, \ttt{scale\_choice\_fixed=0}, \ttt{scale\_choice\_arbitrary=2}?} \ak{Yes, and done!}
% %
% In this last case, the structure functions are saved as arrays not only in
% $Q$ but also $\mu_R$ and $\mu_F$.
% %
% This makes for a slightly slower evaluation compared to the two other
% scale choices.

If \ttt{param\_coefs} is set to \ttt{.true.} (its default) then the
structure functions are computed using the NNLO and N$^3$LO
parametrisations found in
Refs.~\cite{vanNeerven:1999ca,vanNeerven:2000uj,Moch:2004xu,Vermaseren:2005qc,Moch:2008fj,Davies:2016ruz},
which are stated to have a relative precision of a few permille (order
by order) except at particularly small or large values of $x$.
%
Otherwise
the exact versions are used.\footnote{The LO and NLO coefficient
functions are always exact as their expressions are very
compact.}
%
This however means that the initialisation becomes slow (about two
minutes rather than a few seconds).
%
Given the good accuracy of the parametrised coefficient functions,
they are to be preferred for most applications.
% 
% and since the parametrised
% expressions are good to a relative accuracy of $10^{-4}$ for most
% values of $x$, it is recommended to use the parametrised option for
% most applications.
% 
Note that the exact expressions also add to compilation time and need
to be explicitly enabled with the \ttt{--enable-exact-coefs} configure
flag.\footnote{At N$^3$LO they rely on an extended version of
  \ttt{hplog}~\cite{FortranPolyLog}, \texttt{hplog5} version 1.0, that
  is able to handle harmonic polylogarithms up to weight 5.
  %
  However there appears to be a percent-level issue on the coefficient
  functions for $x\lesssim 0.1$, connected with the evaluation of the
  polylogarithms. This is currently under investigation. }

The masses of the electroweak vector bosons are used only to calculate
the weak mixing angle, $\sin^2 \theta_W = 1 - (m_W/m_Z)^2$, which
enters in the neutral-current structure functions.

At this point all the tables that are needed for the structure
functions have been allocated.
%
In order to fill the tables, one first needs to set up the running
coupling and evolve the initial PDF with \ttt{hoppetEvolve}, as
described in
\ifreleasenote
Section~8.2 of Ref.~\cite{Salam:2008qg}.
\else
Section~\ref{sec:vanilla_usage}.
\fi

%
% Care should be taken here such that both the coupling and PDF
% evolution are carried out at the correct perturbative order and with
% mass thresholds as appropriate.
%
With the PDF table filled in the streamlined interface one calls
\begin{lstlisting}
  call InitStrFct(order[, separate_orders])
\end{lstlisting}
specifying the order at which one would like to compute the structure
functions.
%
The logical flag \ttt{separate\_orders} should be set to \ttt{.true.}\ if one
wants access to the individual coefficients of the perturbative
expansion as well as the sum up to some maximum order.\commentgz{the maximum order is ``order''?}
%
With \ttt{scale\_choice\_Q} and \ttt{scale\_choice\_fixed}, the
default of \ttt{.false.}\ causes only the sum over perturbative orders
to be stored.
%
This gives faster evaluations of structure functions because it is
only necessary to interpolate the sum over orders, rather than
interpolate one table for each order.
%
With \texttt{scale\_choice\_arbitrary}, the default is \ttt{.true.},
which is the only allowed option, because separate tables for each
order are required for the underlying calculations.

\subsection{Accessing the Structure Functions}
\label{sec:structure-funcs-access}
At this point the structure functions can be accessed as in the following example
\begin{lstlisting}
  real(dp) :: ff(-6:7), x, Q, muR, muF
  [...]
  call InitStrFct(order_max = 4, scale_choice = scale_choice_Q)
  ff = StrFct(x, Q[, muR] [, muF])
\end{lstlisting}
at the value $x$ and $Q$. \gps{Added explicit scale choice arg in init}
%
With \ttt{scale\_choice\_arbitrary}, the \ttt{muR} and \ttt{muF}
arguments must be provided.
%
With other scale choices, they do not need to be provided, but if they
are then they should be consistent with the original scale choice.
%
The structure functions in this example are stored in the
array \ttt{ff}. The components of this array can be accessed through
the indices
\begin{lstlisting}
  integer, parameter :: iF1Wp= 1   
  integer, parameter :: iF2Wp= 2   
  integer, parameter :: iF3Wp= 3   
  integer, parameter :: iF1Wm=-1   
  integer, parameter :: iF2Wm=-2   
  integer, parameter :: iF3Wm=-3   
  integer, parameter :: iF1Z = 4   
  integer, parameter :: iF2Z = 5   
  integer, parameter :: iF3Z = 6   
  integer, parameter :: iF1EM = -4   
  integer, parameter :: iF2EM = -5   
  integer, parameter :: iF1gZ = 0  
  integer, parameter :: iF2gZ = -6 
  integer, parameter :: iF3gZ = 7  
\end{lstlisting}
For instance one would access the electromagnetic $F_1$ structure
function through \ttt{ff(iF1EM)}. It is returned at the \ttt{order\_max}
that was specified in \ttt{InitStrFct}.
%
The structure functions can also be accessed order by order if the
\ttt{separate\_orders} flag was set to \ttt{.true.} when initialising.
%
They are then obtained as follows
\begin{lstlisting}
  real(dp) :: flo(-6:7), fnlo(-6:7), fnnlo(-6:7), fn3lo(-6:7), x, Q, muR, muF
  [...]
  call InitStrFct(4, .true.)
  flo   = F_LO(x, Q, muR, muF)
  fnlo  = F_NLO(x, Q, muR, muF)
  fnnlo = F_NNLO(x, Q, muR, muF)
  fn3lo = F_N3LO(x, Q, muR, muF)
\end{lstlisting}
The functions return the individual contributions at each order in
$\as$, including the relevant factor of $\as^n$.
%
Hence the sum of \ttt{flo}, \ttt{fnlo}, \ttt{fnnlo}, and
\ttt{fn3lo} would return the full structure function at N$^3$LO as
contained in \ttt{ff} in the example above.
%
Note that in the \ttt{F\_LO} etc.\ calls, the \ttt{muR} and \ttt{muF}
arguments are not optional and that when a prior scale choice has been
made (e.g. \ttt{scale\_choice\_Q}) they are assumed to be consistent
with that prior scale choice.

An example of structure function evaluations using the Fortran~90
interface is to be found in
\repolink{example_f90/structure_functions_example.f90}{example\_f90/structure\_functions\_example.f90}. 
%
\gps{added this}

\subsection{Streamlined interface}
\label{sec:structure-functions-streamlined}
The structure functions can also be accessed through the streamlined
interface, so that they may be called for instance from C/C++. The
functions to be called are very similar to those described above. In
particular a user should call either
\begin{lstlisting}
  call hoppetStartStrFct(order_max)
\end{lstlisting}
with \ttt{order\_max} the maximal
order in $\as$ or, alternatively, the extended version of the interface
\ttt{hoppetStartStrFctExtended} which takes all the same arguments as
\ttt{StartStrFct} described above. One difference is that in order to
use a variable flavour scheme the user should set \ttt{nflav} to a
negative value. After evolving or reading in a PDF, the user then calls
\begin{lstlisting}
  call hoppetInitStrFct(order, separate_orders)
\end{lstlisting}
to initialise the actual structure functions. The structure functions
can then be accessed through the subroutines
\begin{lstlisting}
  real(dp) :: ff(-6:7), flo(-6:7), fnlo(-6:7), fnnlo(-6:7), fn3lo(-6:7), x, Q, muR, muF
  [...]
  call hoppetStrFct(x, Q, muR, muF, ff)        ! Full structure function
  ! or instead, without muR and muF
  call hoppetStrFctNoMu(x, Q, ff)              ! Full structure function
  call hoppetStrFctLO(x, Q, muR, muF, flo)     ! LO term
  call hoppetStrFctNLO(x, Q, muR, muF, fnlo)   ! NLO term
  call hoppetStrFctNNLO(x, Q, muR, muF, fnnlo) ! NNLO term
  call hoppetStrFctN3LO(x, Q, muR, muF, fn3lo) ! N3LO term
\end{lstlisting}
The C++ header contains indices for the structure functions and scale
choices, which are all in the \ttt{hoppet} namespace.
%
\begin{lstlisting}
  const int iF1Wp = 1+6;
  const int iF2Wp = 2+6;
  const int iF3Wp = 3+6;
  const int iF1Wm =-1+6;
  const int iF2Wm =-2+6;
  const int iF3Wm =-3+6;
  const int iF1Z  = 4+6;
  const int iF2Z  = 5+6;
  const int iF3Z  = 6+6;
  const int iF1EM =-4+6;
  const int iF2EM =-5+6;
  const int iF1gZ = 0+6;
  const int iF2gZ =-6+6;
  const int iF3gZ = 7+6;

  const int scale_choice_fixed     = 0;
  const int scale_choice_Q         = 1;
  const int scale_choice_arbitrary = 2;
\end{lstlisting}
Note that in C++ the structure function indices start from 0 and that the C++
array that is to be passed to functions such as \ttt{hoppetStrFct}
would be defined as \ttt{double ff[14]}.

An example of structure function evaluations using the C++ version of
the streamlined interface is to be found in
\repolink{example_f77/cpp_structure_functions_example.cc}{example\_f77/cpp\_structure\_functions\_example.cc}.
%
\gps{added this}



%======================================================================
%======================================================================

 	

%%% Local Variables:
%%% TeX-master: "HOPPET-v1.3-release.tex"
%%% End:
