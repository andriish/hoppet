\section{Perturbative evolution in QCD}
\label{sec:pqcd}
First of all we set up the notation and
conventions that are used throughout \hoppet. The DGLAP
equation for a non-singlet parton distribution reads
\begin{equation}
  \label{eq:dglap-ns}
  \frac{\partial q(x,Q^2)}{\partial \ln Q^2} = 
\frac{\aq}{2\pi}\int_x^1 \frac{dz}{z}
  P(z,\aq) q\lp \frac{x}{z},Q^2\rp \equiv 
\frac{\aq}{2\pi}  P(x,\aq) \otimes q\lp x,Q^2\rp \ .
\end{equation}
The related variable $t\equiv \ln Q^2$ is also used
in various places in \hoppet.
The splitting functions in eq.~(\ref{eq:dglap-ns})
are known exactly up to NNLO in the 
unpolarised case \cite{Furmanski:1980cm,Curci:1980uw,NNLO-NS,NNLO-singlet}, and approximately at N$^3$LO~\cite{Gracey:1994nn,Davies:2016jie,Moch:2017uml,Gehrmann:2023cqm,Falcioni:2023tzp,Gehrmann:2023iah,McGowan:2022nag,NNPDF:2024nan,Moch:2021qrk,Falcioni:2023luc,Falcioni:2023vqq,Moch:2023tdj,Falcioni:2024xyt,Falcioni:2024qpd}:
\begin{equation}
  \label{eq:dpdf}
   P(z,\aq)=P^{(0)}(z)+\frac{\aq}{2\pi}P^{(1)}(z)+
\lp \frac{\aq}{2\pi} \rp^2 P^{(2)}(z) 
+
\lp \frac{\aq}{2\pi} \rp^3 P^{(3)}(z) \ ,
\end{equation}
and up to NNLO \cite{Mertig:1995ny,Vogelsang:1996im,Moch:2014sna,Moch:2015usa,Blumlein:2021enk,Blumlein:2021ryt} in the polarised case.
The generalisation to the singlet case is straightforward, as it
is 
%the generalisation of eq.~(\ref{eq:dglap-ns}) 
to the case of time-like evolution\footnote{
The general structure of the relation between space-like
and time-like evolution and splitting functions
 has been investigated in \cite{Furmanski:1980cm,Curci:1980uw,Stratmann:1996hn,Dokshitzer:2005bf,Mitov:2006ic,Basso:2006nk,Dokshitzer:2006nm,Beccaria:2007bb}.\ak{This list may also need some update..}}, 
relevant for example for fragmentation function analysis,
where NNLO results
are also available \cite{Mitov:2006ic,Moch:2007tx,Almasy:2011eq}.


As with the splitting functions, all perturbative quantities in
\hoppet are defined to be coefficients of powers of $\as/2\pi$. The one
exception is the $\beta$-function coefficients of the running
coupling equation:
\begin{equation}
  \label{eq:as-ev}
  \frac{d\as}{d\ln Q^2} = \beta\lp \aq\rp = -\as (\beta_0\as +
  \beta_1\as^2 + 
  \beta_2\as^3 + 
  \beta_3\as^4) \ .
\end{equation}

The evolution of the strong coupling and the parton distributions can
be performed in both the fixed flavour-number scheme (FFNS) and the 
variable flavour-number scheme (VFNS). In the VFNS case we 
need the matching conditions between the effective
theories with $n_f$ and $n_{f}+1$ light flavours for both the strong 
coupling $\aq$ and the parton distributions at the heavy quark
mass threshold $m_h^2$.

These matching conditions for the parton distributions
receive non-trivial contributions at higher orders. In the $\MSbar$
(factorisation) scheme, for example,
carrying out the matching at a scale equal to the heavy-quark mass
these begin at NNLO:\footnote{In
  a general scheme they would start at NLO.} %
for light quarks $q_{l,i}$ of flavour $i$ 
(quarks that are considered massless
below the heavy quark mass threshold $m_h^2$) the matching between
their values in the $n_f$ and
$n_f+1$ effective theories reads\ak{Do we need a footnote pointing out that this structure is not entirely clear from the literature?}:
%\begin{equation}
%\label{eq:lp-nf1}
%  q_{l,i}^{\,(n_f+1)}(x,m_h^2) \: = \:  q_{l,i}^{\,(\nf}(x,m_h^2) +
%\lp \frac{\alpha_s(m_h^2)}{2\pi} \rp^2
%   A^{\rm ns,(2)}_{qq,h}(x) \otimes
%  q_{l,i}^{\, (\nf}(x,m_h^2) \ ,
%\end{equation}
\begin{align}
\label{eq:lp-nf1}
  q_{l,i}^{\,(n_f+1)}(x,m_h^2) + q_{l,-i}^{\,(n_f+1)}(x,m_h^2)  & =   q_{l,i}^{\,(\nf}(x,m_h^2) + q_{l,-i}^{\,(\nf}(x,m_h^2) \notag \\ &+
   A^{\rm NS,+}_{qq,h}(x) \otimes \left(
   q_{l,i}^{\, (\nf}(x,m_h^2) + q_{l,-i}^{\, (\nf}(x,m_h^2)\right) \notag\\
   & + \frac{1}{n_f} \Big\{A^{\rm PS}_{qq,h}(x) \otimes \Sigma^{\, (\nf}(x,m_h^2) \notag\\
   & + A^{\rm S}_{qg,h}(x) \otimes g^{\, (\nf}(x,m_h^2)\Big\} \ , \notag \\
  q_{l,i}^{\,(n_f+1)}(x,m_h^2) - q_{l,-i}^{\,(n_f+1)}(x,m_h^2)  & =   q_{l,i}^{\,(\nf}(x,m_h^2) - q_{l,-i}^{\,(\nf}(x,m_h^2) \notag \\ &+
   A^{\rm NS,-}_{qq,h}(x) \otimes \left(
   q_{l,i}^{\, (\nf}(x,m_h^2) - q_{l,-i}^{\, (\nf}(x,m_h^2)\right) \ ,
\end{align}
where $i = 1,\ldots n_f$, while for the gluon distribution, the heavy
quark PDF $q_h$, and the singlet PDF $\Sigma(x,Q^2)$ (defined in
Table~\ref{eq:diag_split}) one has :
\begin{align}
\label{eq:hp-nf1}
  g^{(n_f+1)}(x,m_h^2)  &=
    g^{\, (\nf}(x,m_h^2) +
    A_{\rm gq,h}^{\rm S}(x) \otimes \Sigma^{(\nf}(x,m_h^2) +
    A_{\rm gg,h}^{\rm S}(x) \otimes g^{(\nf}(x,m_h^2) \ ,
  \nn \\[0.3cm]
  (q_h+\bar{q}_{h})^{(n_f+1)}(x,m_h^2)  &=
  A_{\rm hq}^{\rm S}(x)\otimes \Sigma^{(\nf}(x,m_h^2) 
  + A_{\rm hg}^{\rm S}(x)\otimes g^{(\nf}(x,m_h^2)\ ,  \nn \\[0.3cm]
  \Sigma^{(n_f+1)}(x,m_h^2)  &= \Sigma^{\, (\nf}(x,m_h^2) + \left[ A^{\rm NS,+}_{qq,h}(x) + A^{\rm PS}_{qq,h}(x) + A_{\rm hq}^{\rm S}(x)\right] \otimes \Sigma^{(\nf}(x,m_h^2) \nn \\
  & + \left[ A^{\rm S}_{qg,h}(x) + A_{\rm hg}^{\rm S}(x) \right] \otimes g^{(\nf}(x,m_h^2)
\end{align}
with $q_h=\bar{q}_h$. Up to N$^3$LO the matching coefficients have the
following expansions in $\alpha_s$
\begin{align}
  A^{\rm NS,\pm}_{qq,h}(x) & = \lp \frac{\alpha_s(m_h^2)}{2\pi} \rp^2
  A^{\rm NS,\pm,(2)}_{qq,h}(x) + \lp \frac{\alpha_s(m_h^2)}{2\pi} \rp^3
  A^{\rm NS,\pm,(3)}_{qq,h}(x) \ , \nn \\
  A^{\rm S}_{gk,h}(x) & = \lp \frac{\alpha_s(m_h^2)}{2\pi} \rp^2
  A^{\rm S,(2)}_{gk,h}(x) + \lp \frac{\alpha_s(m_h^2)}{2\pi} \rp^3
  A^{\rm S,(3)}_{gk,h}(x), \quad k=q,g \ , \nn \\
  %A^{\rm S}_{gg,h}(x) & = \lp \frac{\alpha_s(m_h^2)}{2\pi} \rp^2
  %A^{\rm S,(2)}_{gg,h}(x) + \lp \frac{\alpha_s(m_h^2)}{2\pi} \rp^3
  %A^{\rm S,(3)}_{gg,h}(x) \ , \nn \\
  A^{\rm S}_{hk}(x) & = \lp \frac{\alpha_s(m_h^2)}{2\pi} \rp^2
  A^{\rm S,(2)}_{hk}(x) + \lp \frac{\alpha_s(m_h^2)}{2\pi} \rp^3
  A^{\rm S,(3)}_{hk}(x), \quad k=q,g \ , \nn \\
  %A^{\rm S}_{hg}(x) & = \lp \frac{\alpha_s(m_h^2)}{2\pi} \rp^2
  %A^{\rm S,(2)}_{hg}(x) + \lp \frac{\alpha_s(m_h^2)}{2\pi} \rp^3
  %A^{\rm S,(3)}_{hg}(x) \ , \nn \\
  A^{\rm PS}_{qq,h}(x) & = \lp \frac{\alpha_s(m_h^2)}{2\pi} \rp^3
  A^{\rm PS,(3)}_{qq,h}(x) \ , \nn \\
  A^{\rm S}_{qg,h}(x) & = \lp \frac{\alpha_s(m_h^2)}{2\pi} \rp^3
  A^{\rm S,(3)}_{qg,h}(x)
\end{align}
At $\mathcal{O}(\alpha_S^2)$ we have that $A^{\rm NS,+}_{qq,h}(x) =
A^{\rm NS,-}_{qq,h}(x)$ whereas they start to differ at
$\mathcal{O}(\alpha_S^3)$. The NNLO matching coefficients were
computed in \cite{NNLO-MTM}\footnote{The authors are thanked for the
code corresponding to the calculation.} and the N$^3$LO matching
coefficients
in~\cite{Bierenbaum:2009mv,Ablinger:2010ty,Kawamura:2012cr,Blumlein:2012vq,ABLINGER2014263,Ablinger:2014nga,Ablinger:2014vwa,Behring:2014eya,Ablinger:2019etw,Behring:2021asx,Fael:2022miw,Ablinger:2023ahe,Ablinger:2024xtt,BlumleinCode}\footnote{We
thank Johannes Bl\"umlein for sharing the code in
Ref.~\cite{BlumleinCode} with us, which also contains code associated
with Refs.~\cite{Ablinger:2024xtt,Fael:2022miw}.
%
}
%
Notice that the above
conditions will lead to small discontinuities of the PDFs in its
evolution in $Q^2$, which are cancelled by similar matching terms in
the coefficient functions resulting in continuous physical
observables. In particular, the heavy quark PDFs start from a non-zero
value at threshold at NNLO, which sometimes can even be negative.

The corresponding NNLO relation for the matching of the $\MSbar$
coupling constant at the heavy quark threshold $m^2_h$ is given by 
\begin{equation}
\label{eq:as-nf1}
  \as^{\, (n_f+1)}(m_h^2) \: = \:
  \as^{\, (\nf} (m_h^2) +   C_2 \lp \frac{\as^{\, (\nf} (m_h^2)}{2\pi} \rp^3+   C_3 \lp \frac{\as^{\, (\nf} (m_h^2)}{2\pi} \rp^4
   \:\: ,
\end{equation}
where the matching coefficients $C_2$ and $C_3$ were computed in
\cite{Chetyrkin:1997sg,Chetyrkin:1997un}.
%
The value and the form of the matching coefficients in
eqs.~(\ref{eq:lp-nf1},\ref{eq:hp-nf1}) depend on the scheme used for
the quark masses; by default in \hoppet quark masses are taken to be
pole masses, though the option exists for the user to supply and have
thresholds crossed at $\MSbar$ masses, but only up to NNLO. We note
that in the current implementation in \hoppet the matching can only be
performed at the matching point that corresponds to the quark masses
themselves.

Both evolution and threshold matching preserve the momentum sum rule
\begin{equation}
  \int_0^1 dx~x \lp \Sigma(x,Q^2)+g(x,Q^2)\rp =1 \,,
\end{equation}
and valence sum rules
\begin{equation}
  \int_0^1 dx\, \left[q(x,Q^2)-{\bar q}(x,Q^2) \right] = \left\{ 
    \begin{array}{ll}
      1, & \text{for } q = d \text{ (in proton)}\\
      2, & \text{for } q = u \text{ (in proton)}\\
      0, & \text{other flavours}
    \end{array}
    \right.
\end{equation}
as long as they hold at the initial scale (occasionally not the case,
\eg in modified LO sets for Monte Carlo
generators~\cite{Sherstnev:2008dm}).

The default basis for the PDFs, called the \ttt{human} 
representation in \hoppet, is such that 
 the entries in an array
\ttt{pdf(-6:6)} of PDFs correspond to:
\bea 
\bar t={-6} \ ,  \bar b={-5} \ ,  \bar c={-4}
\ , \nn   \bar s&=&{-3} \ , \nn  \bar u={-2} \ , \nn
 \bar d={-1} \ , \\  g&=&{0} \ , \\ \nn   d={1} \ , \nn  u={2} 
\ , \nn  
s={3} \ , \nn   c&=&{4} \ , \nn b={5} \ , \nn  t={6} \ . \nn 
\eea
 This representation is the
same as that used in the \ttt{LHAPDF} library \cite{LHAPDF}. 
However, this representation leads
to a complicated form of the evolution equations.
The splitting matrix can be simplified considerably (made diagonal
except for a $2\times2$ singlet block) by switching to a different
flavour representation, which is named
the \ttt{evln} representation, for the PDF set, as explained in detail in
\cite{vanNeerven:1999ca,vanNeerven:2000uj}. This representation
is described in Table \ref{eq:diag_split}.

In the {\tt evln} basis, 
the gluon evolves coupled to the singlet  PDF $\Sigma$,
and all non-singlet PDFs evolve independently.
Notice that the representations of the PDFs
are preserved under linear operations, so in particular
they are preserved under DGLAP evolution.
The conversion from the \ttt{human} to the \ttt{evln}
representations of PDFs requires that the number of
active quark flavours $n_f$ be specified by the user, as described in
\ifreleasenote
Section~5.1.2 of Ref.~\cite{Salam:2008qg}.
\else
Section~\ref{sec:evln-rep}.
\fi

\begin{table}
\begin{center}
\begin{tabular}{|r | c | l |}
\hline
     i & \mbox{name} & $q_i$ \\ \hline
     $ -6\ldots-(n_f+1)$ & $q_i$ & $q_i$\\
     $-n_f\ldots -2$ & $q_{\mathrm{NS},i}^{-}$ & 
$(q_i -  {\bar q}_i) - (q_1 - {\bar q}_1)$\\
      -1           & $q_{\mathrm{NS}}^{V}$ & 
$\sum_{j=1}^{n_f} (q_j -  {\bar q}_j)$\\
       0           & g & \textrm{gluon} \\
       1           & $\Sigma$ & $\sum_{j=1}^{n_f} (q_j +  {\bar q}_j)$\\
     $2\ldots n_f$ & $q_{\mathrm{NS},i}^{+}$ &
$ (q_i +  {\bar q}_i) - (q_1 + {\bar q}_1)$\\
      $(n_f+1)\ldots6$ & $q_i$ & $q_i$ \\
\hline
\end{tabular}
\caption{}{\label{eq:diag_split} The evolution representation 
(called \ttt{evln} in \hoppet)
of PDFs with $n_f$ active quark flavours
in terms of the \ttt{human} representation.}  
\end{center}
\end{table}

In \hoppet unpolarised DGLAP evolution is available up to N$^3$LO
in the $\MSbar$ scheme, while for the DIS scheme
only evolution up to NLO is available, but without the NLO heavy-quark
threshold matching conditions. For polarised evolution up to NLO only
the $\MSbar$ scheme is available. The variable \ttt{factscheme}
takes different values for each factorisation scheme:
\begin{center}
  \begin{tabular}{|c|l|}\hline
    \ttt{factscheme} & Evolution\\[2pt]\hline
    1 & unpolarised $\MSbar$ scheme\\[2pt]\hline
    2 & unpolarised DIS scheme\\[2pt]\hline
    3 & polarised $\MSbar$ scheme\\\hline
  \end{tabular}
\end{center}
Note that mass thresholds are currently
missing in the DIS scheme.
