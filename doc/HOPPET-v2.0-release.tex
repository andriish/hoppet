\documentclass[preprint,1p,a4paper,11pt]{elsarticle}

\usepackage{amsmath}
\usepackage{booktabs,cellspace}
\usepackage{amssymb,amssymb,mathtools}
\usepackage{url}
\usepackage{xspace}
\usepackage{graphicx}
\usepackage{a4wide}
\usepackage{color}
\usepackage{listings}
\usepackage{hyperref}
%\usepackage[style=numeric,citestyle=numeric-comp,sorting=ynt,defernumbers=true,backend=biber]{biblatex}

% \textwidth 21cm
% \linewidth 21cm
% \columnwidth 21cm
% 
%\marginparwidth 0 in

%\textheight 22.5cm \textwidth 16cm
%\oddsidemargin 0.0cm \evensidemargin 0.0cm
%\topmargin -2.3cm  % for nikhef
%\topmargin -0.5cm  % for hep-ph
\def\CPP{{C\nolinebreak[4]\hspace{-.05em}\raisebox{.4ex}{\tiny\textbf{++}}}}
\newcommand{\be}{\begin{equation}}
\newcommand{\ee}{\end{equation}}
\newcommand{\bea}{\begin{eqnarray}}
\newcommand{\eea}{\end{eqnarray}}
\newcommand{\bi}{\begin{itemize}}
\newcommand{\ei}{\end{itemize}}
\newcommand{\ben}{\begin{enumerate}}
\newcommand{\een}{\end{enumerate}}
\newcommand{\la}{\left\langle}
\newcommand{\ra}{\right\rangle}
\newcommand{\lc}{\left[}
\newcommand{\rc}{\right]}
\newcommand{\lp}{\left(}
\newcommand{\rp}{\right)}
\newcommand{\aq}{\alpha_s\left( Q^2 \right)}
\newcommand{\amz}{\alpha_s\left( M_Z^2 \right)}
\newcommand{\aqq}{\alpha_s \left( Q^2_0 \right)}
\newcommand{\aqz}{\alpha_s \left( Q^2_0 \right)}
\newcommand{\nf}{n_f)}
\newcommand{\nn}{\nonumber}
\newcommand{\nin}{\noindent}

\newcommand{\dy}{\ttt{dy}}
\newcommand{\dlnlnQ}{\ttt{dlnlnQ}}
\newcommand{\bq}{\boldsymbol{q}}
\newcommand{\mus}{\;\mu\mathrm{s}}
\newcommand{\ms}{\;\mathrm{ms}}
\newcommand{\GeV}{\;\mathrm{GeV}}
\newcommand{\TeV}{\;\mathrm{TeV}}
\newcommand{\as}{\alpha_s}

% author comments
\definecolor{darkgreen}{rgb}{0,0.6,0}
\newcommand{\comment}[1]{{\color{red}\textbf{[#1]}}}
\newcommand{\commentgz}[1]{{\color{red} [\it GZ:  #1}]}
\newcommand{\commentpn}[1]{{\color{magenta} [\it PN:  #1}]}
\newcommand{\gps}[1]{{\textcolor{darkgreen}{\comment{#1}$_\text{GPS}$}}}
\newcommand{\ak}[1]{{\textcolor{blue}{\comment{#1}$_\text{AK}$}}}


\newcommand{\eg}{e.g.\ }
\newcommand{\ie}{i.e.\ }
\newcommand{\cf}{cf.\ }
\newcommand{\MSbar}{\overline{\mathrm{MS}}}
\newcommand{\hoppet}{\textsc{hoppet}\xspace}
\newcommand{\ttt}[1]{\texttt{#1}}
\newcommand{\order}[1]{{\cal O}\left(#1\right)}
\newcommand{\fn}{\scriptsize}

\newcommand{\AllDGLAP}{Botje,Schoeffel:1998tz,Pegasus,Pascaud:2001bi,Weinzierl:2002mv,Cafarella:2003jr,Cafarella:2005zj,Cafarella:2008du,GuzziThesis,nnpdf,Kosower:1997hg,Ratcliffe:2000kp}

\newcommand{\myparagraph}[1]{\paragraph{#1.}}

% Define special colors
\definecolor{comment}{rgb}{0,0.3,0}
\definecolor{identifier}{rgb}{0.0,0,0.3}


%% \lstset{% general command to set parameter(s)
%%   columns=fullflexible
%%   basicstyle=\tt, % print whole listing small
%%   keywordstyle=\color{black},
%%   % underlined bold black keywords
%%   identifierstyle=, % nothing happens
%%   commentstyle=\color{comment}, % white comments
%%   stringstyle=\ttfamily, % typewriter type for strings
%%   showstringspaces=false} % no special string spaces

\lstset{language=Fortran}
\lstset{
  columns=flexible,
  basicstyle=\tt\footnotesize,
  keywordstyle=,
  identifierstyle=\color{black},
  commentstyle=\tt\color{comment},
  mathescape=true,
  escapebegin=\color{comment},
  showstringspaces=false,
  keepspaces=true
}

\geometry{margin=25mm,
    headheight=110pt,
    footskip=30pt
}
\newcommand{\repolink}[2]{\href{https://github.com/hoppet-code/hoppet/blob/master/#1}{\ttt{#2}}}
\newcommand{\masterlink}[1]{\repolink{#1}{#1}}
\biboptions{sort&compress}

% We need slightly different text in the manual and in the release note.
\newif\ifreleasenote
\releasenotetrue % comment out to hide answers

\journal{European Physical Journal C}

\begin{document}
\begin{frontmatter}
\begin{flushright}
CERN-TH-2023-237,  MPP-2023-285, OUTP-23-15P
\end{flushright}
\title{\hoppet{} {\tt v2.0.0} release note}

\author[1]{Alexander Karlberg}\ead{alexander.karlberg@cern.ch}
\author[2]{Paolo Nason}\ead{paolo.nason@mib.infn.it}
\author[3,4]{Gavin Salam}\ead{gavin.salam@physics.ox.ac.uk}
\author[5,6]{Giulia Zanderighi}\ead{zanderi@mpp.mpg.de}
\author[3]{Fr\'ed\'eric Dreyer}\ead{frederic.dreyer@gmail.com}

\affiliation[1]{organization={CERN, Theoretical Physics Department}, postcode={CH-1211} ,city={Geneva 23}, country={Switzerland}}
\affiliation[2]{organization={INFN, Sezione di Milano-Bicocca, and Universita di Milano-Bicocca, \mbox{Piazza della Scienza 3}}, postcode={20126} ,city={Milano}, country={Italy}}
\affiliation[3]{organization={Rudolf Peierls Centre for Theoretical Physics, Clarendon Laboratory, Parks Road}, postcode={OX1 3PU} ,city={Oxford}, country={UK}}
\affiliation[4]{organization={All Souls College}, postcode={OX1 4AL} ,city={Oxford}, country={UK}}
\affiliation[5]{organization={Max-Planck-Institut fur Physik, Boltzmannstr. 8}, postcode={85748} ,city={Garching}, country={Germany}}
\affiliation[6]{organization={Physik-Department, Technische Universitat Munchen, James-Franck-Strasse 1}, postcode={85748} ,city={Garching}, country={Germany}}

\begin{abstract}
  We document the three main new features in release 2.0.0 of the
  \hoppet parton distribution function evolution code, specifically
  support for QED evolution to an accuracy phenomenologically
  equivalent to NNLO QCD, for N$^3$LO QCD evolution in the variable
  flavour number scheme, and for the determination of hadronic
  structure functions for massless quarks up to N$^3$LO. Additionally
  we describe a new Python interface and CMake build option.
\end{abstract}

\begin{keyword}
  Perturbative QCD \sep DIS \sep DGLAP \sep QED
%% keywords here, in the form: keyword \sep keyword

%% PACS codes here, in the form: \PACS code \sep code

%% MSC codes here, in the form: \MSC code \sep code
%% or \MSC[2008] code \sep code (2000 is the default)

\end{keyword}

\end{frontmatter}
\tableofcontents
\section{Introduction}

\hoppet is a parton distribution function (PDF) evolution code written
in Fortran~95\footnote{This release of \hoppet makes use of features
introduced in Fortran~2003, in particular its \ttt{abstract
  interface}, and hence a Fortran~2003 compliant compiler is now needed.},
with interfaces also for C/\CPP{} and earlier dialects of Fortran.
%
It offers both a high-level PDF evolution interface and user-access to
lower-level functionality for operations such as convolutions of
coefficient functions and PDFs, and is designed to provide flexible and
fast evolution.

This release note documents three major additions to \hoppet, made
available as part of release~2.0.0: (1) support for QED evolution,
originally developed as part of the LuxQED project for the evaluation
of the photon density inside a proton and its extension to lepton
distributions in the
proton~\cite{Manohar:2016nzj,Manohar:2017eqh,Buonocore:2020nai,Buonocore:2021bsf};
%
(2) support for QCD evolution at N$^3$LO in the (zero mass) variable
flavour number scheme (VFNS)~\cite{Buza:1996wv};
%
and (3) support for the
determination of massless hadronic structure functions, as initially developed
for calculations of vector-boson fusion cross
sections~\cite{Cacciari:2015jma,Dreyer:2016oyx,Dreyer:2018qbw,Dreyer:2018rfu}.

This release also includes a range of other smaller additions relative
to the original 1.1.0 release documented in \cite{Salam:2008qg}, in
particular the addition of a new Python interface and CMake build
option.
%
Unified documentation of the whole \hoppet package is part of the
distribution at \url{https://github.com/hoppet-code/hoppet} in the
\masterlink{doc/} directory.
%
Details of the other changes since release 1.1.0 can be found in the
\masterlink{NEWS} and \masterlink{ChangeLog} files from the
repository.



% ======================================================================
\section{Evolution including QED contributions}
\label{sec:qed-evolution}

The combined QED+QCD evolution, as implemented in \hoppet since
version 1.3.0 (and earlier in a dedicated \ttt{qed} branch), was first
described in
Refs.~\cite{Manohar:2016nzj,Manohar:2017eqh,Buonocore:2020nai,Buonocore:2021bsf}.
%
The determination of which contributions to include follows a
consistent approach based on the so-called ``phenomenological'' counting
scheme.
%
Within this scheme, one considers $\alpha$ to be of order $\as^2$, and takes the
photon (lepton) PDF to be of order $\alpha L$ ($\alpha^2 L^2$), where
$L$ is the logarithm of a ratio of the factorisation scale to a typical hadronic scale and is considered to be of order $L \sim 1/\as$. In contrast, quark and gluon
PDFs are considered to be of order $(\as L)^n={\cal O}(1)$.\footnote{The
  above counting is to be contrasted with a ``democratic'' scheme, in
  which one considers $\alpha \sim \as \sim 1/L$ and the aim is to maintain
  the same loop order across all couplings and splitting functions,
  regardless of the relative numbers of QCD and QED couplings that
  they involved.}
%
From this point of view, NNLO (3-loop) QCD evolution provides control of
terms of order up to $\as^{n+2} L^n \sim \as^2$.
%
To achieve a corresponding accuracy when including QED contributions,
\hoppet has been extended to account for
\begin{enumerate}
\item \label{item:qed1} 1-loop QED splitting functions~\cite{Roth:2004ti}, which first
  contribute at order $\alpha L \sim \as$, i.e.\ count as NLO QCD
  corrections;
  
\item \label{item:qed2} 1-loop QED running coupling, including lepton and quark
  thresholds, which first contributes at order $\alpha^2 L^2 \sim
  \as^2$, i.e. like NNLO QCD; 
  
\item \label{item:qed3} 2-loop mixed QCD-QED splitting
  functions~\cite{deFlorian:2015ujt},
  %\footnote{One-loop triple collinear splitting functions with photons have been computed in refs~\cite{Sborlini:2014mpa,Sborlini:2014kla}};,
  which first contribute at order
  $\alpha \as L \sim \as^2$, i.e.\ count as NNLO QCD corrections;

\item \label{item:qed4} optionally, the 2-loop pure QED $P_{\ell q}$ splitting
  function~\cite{deFlorian:2016gvk}, which brings absolute accuracy
  $\alpha^2 L\sim \as^3$ to the lepton distribution (which starts at
  $\alpha^2 L^2 \sim \as^2$).\footnote{The implementation of the 2-loop $P_{\ell q}$ splitting
  function in the hoppet code (\ttt{Plq\_02} in the code) was carried out by Luca Buonocore.}
  %
  In an absolute counting of accuracy, this is not needed.
  % 
  However, if one wants lepton distributions to have the same relative
  NLO accuracy as the photon distribution, it should be
  included.
\end{enumerate}
%
The code could be extended systematically to aim at a higher
accuracy. For instance, if one wished to reach N$^3$LO accuracy in the
phenomenological counting, one would need to include 3-loop mixed
QCD-QED splitting functions at order $\alpha \as^2$ which contribute
at order $\alpha \as^2 L \sim \as^3$, which are currently not
available, and the well-known 2-loop QED running coupling at order $\alpha \alpha_s$ which contribute at order $\alpha^2 \as L^2 \sim \as^3$.
%
%\comment{Discuss what is needed to go effectively to N3LO? $\alpha^2$
%  splitting exists, $\alpha\as^2$ does not?}
%


\subsection{Implementation of the QED extension}

\myparagraph{QED coupling}

A first ingredient is the setup of the QED coupling object, defined in
\ttt{module qed\_coupling}:
\begin{lstlisting}
  type qed_coupling 
     real(dp) :: m_light_quarks
     real(dp) :: mc, mb, mt
     integer  :: n_thresholds
     integer  :: nflav(3, 0:n_thresholds) ! first index: 1 = nleptons, 2=ndown, 3=nup
     real(dp) :: thresholds(0:n_thresholds+1)
     real(dp) :: b0_values(n_thresholds)
     real(dp) :: alpha_values(n_thresholds)
  end type qed_coupling
\end{lstlisting}
%
This is initialized through a call to 
\begin{lstlisting}
  subroutine InitQEDCoupling(coupling, m_light_quarks, &
   &                           m_heavy_quarks(4:6) [,value_at_scale_0])
    type(qed_coupling), intent(out) :: coupling
    real(dp),           intent(in)  :: m_light_quarks, m_heavy_quarks(4:6)
    real(dp), optional, intent(in)  :: value_at_scale_0 ! defaults to alpha_qed_scale_0
    [...]
  end subroutine InitQEDCoupling
\end{lstlisting}

It initialises the parameters relevant to the QED coupling and its
running.  The electromagnetic coupling at scale zero is set by default
to its PDG~\cite{ParticleDataGroup:2022pth} value, unless the optional
argument \ttt{value\_at\_scale\_0} is provided, in which case the
latter is taken as the value of the QED coupling at zero momentum.

The running is performed at leading order level using seven
thresholds: a common effective mass for the three light quarks
(\ttt{m\_light\_quarks}), the three lepton masses (hard-coded to their
PDG values~\cite{ParticleDataGroup:2022pth} in the
\ttt{src/qed\_coupling.f90} file), and the three masses of the heavy
quarks (\ttt{m\_heavy\_quarks(4:6)}).
%
The common value of the light quark masses is used to mimic the
physical evolution in the region $0.1\GeV \lesssim \mu \lesssim
1\GeV$, which involves hadronic states.
%
Using a value of $0.109\GeV$ generates $\MSbar$ QED coupling values at
the masses of the $\tau$ ($1/133.458$)and $Z$-boson ($1/127.952$) that
agree with the PDG ones ($1/(133.471\pm0.007)$ and $1/(127.951\pm0.009)$)
with a relative $\sim 10^{-4}$ accuracy.
%
%Lepton masses are hard-coded to their PDG
%values~\cite{ParticleDataGroup:2022pth} in the
%\ttt{src/qed\_coupling.f90} file.
%
%The heavy-quark masses \ttt{quark\_masses(4:6)} are instead required as
%non-optional arguments.

The quark and lepton masses are used to set all thresholds where the
fermion content changes. This information is then used to set
the array \ttt{nflav(3,0:n\_thresholds)}, where \ttt{nflav(1:3,i)}
contains an integer array with the number of leptons, down, and up
quarks at a given $Q^2$ such that \ttt{threshold(i-1)}$<Q^2<$
\ttt{threshold(i)}.
%
The \ttt{threshold(1:7)} entries are active thresholds, while
\ttt{threshold(0)} is set to zero and \ttt{threshold(8)} to
$10^{200}$. The integer array \ttt{nflav(1:3,0:n\_thresholds)} is then used to compute the
$\beta_{0,\rm QED}$ function \ttt{b0\_values(1:n\_thresholds)} at the seven threshold values and this is
finally used to compute the value of the QED coupling
\ttt{alpha\_values(1:n\_thresholds)} at the threshold values.
%
The function \ttt{Delete(qed\_coupling)} is also provided for
consistency, although in this case it does nothing.
%
After this initialization, the function \ttt{Value(qed\_coupling,mu)} returns the QED coupling at
scale $\mu$.

\myparagraph{QED splitting matrices}

The QED splitting matrices are stored in the object

\begin{lstlisting}
  type qed_split_mat
     type(qed_split_mat_lo)   :: lo
     type(qed_split_mat_nlo)  :: nlo
     type(qed_split_mat_nnlo) :: nnlo
  end type qed_split_mat
\end{lstlisting}
defined in \ttt{qed\_objects.f90}. 
This contains the LO, NLO and NNLO splitting matrices 
\begin{lstlisting}
  ! a leading-order splitting matrix (multiplies alpha/2pi)
  type qed_split_mat_lo
     type(grid_conv) :: Pqq_01, Pqy_01, Pyq_01, Pyy_01
     integer         :: nu, nd, nl, nf
  end type qed_split_mat_lo

  ! a NLO splitting matrix (multiplies (alpha alpha_s)/(2pi)^2)
  type qed_split_mat_nlo
     type(grid_conv) :: Pqy_11, Pyy_11, Pgy_11
     type(grid_conv) :: Pqg_11, Pyg_11, Pgg_11
     type(grid_conv) :: PqqV_11, PqqbarV_11, Pgq_11, Pyq_11
     integer         :: nu, nd, nl, nf
  end type qed_split_mat_nlo

  ! a NNLO splitting matrix (multiplies (alpha/(2pi) )^2) 
  ! contains only Plq splitting!
  type qed_split_mat_nnlo 
     type(grid_conv) :: Plq_02
     integer         :: nu, nd, nl, nf
  end type qed_split_mat_nnlo
\end{lstlisting}     

Above \ttt{y} denotes a photon and the pairs of integers \ttt{01},
\ttt{11} and \ttt{02} denote order in the QCD and QED couplings,
respectively. Besides the number of quarks \ttt{nf}, these splitting
matrices also need the number of up-type (\ttt{nu}) and down-type
quarks (\ttt{nd}) separately, and the number of leptons (\ttt{nl}).
%
Note that the splitting functions of order $\alpha$ (i.e.\ \ttt{01})
for the leptons are simply obtained from the ones
involving quarks by adjusting colour factors and couplings.

A call to the subroutine 
\begin{lstlisting}     
  subroutine InitQEDSplitMat(grid, qed_split)
    use qed_splitting_functions
    type(grid_def),      intent(in)    :: grid
    type(qed_split_mat), intent(inout) :: qed_split
    [...]
  end subroutine InitQEDSplitMat
\end{lstlisting}     
initializes the \ttt{qed\_split\_mat} object \ttt{qed\_split} and sets all QED
splitting functions on the given \ttt{grid}.
%
The above QED objects can be used for any sensible value of the
numbers of flavours, on the condition that one first registers
the current number of flavours with a call to
\begin{lstlisting}
  QEDSplitMatSetNf(qed_split, nl, nd, nu)
\end{lstlisting}
where \ttt{nl}, \ttt{nd} and \ttt{nu} are respectively the current
numbers of light leptons, down-type and up-type quarks.
%
In practice, this is always handled internally by the QED-QCD
evolution routines, based on the thresholds encoded in the QED
coupling.\footnote{While the QCD splitting functions are initialised
  and stored separately for each relevant value of $n_f$, in the QED
  case the parts that depend on the numbers of flavours are separated
  out.
  %
  Only when the convolutions with PDFs are performed are the relevant
  $n_f$ and electric charge factors included.}
%
The one situation where a user would need to call this routine
directly is if they wish to manually carry out convolutions of the
QED splitting functions with a PDF.

Subroutines \ttt{Copy} and \ttt{Delete} are also provided for the
\ttt{qed\_split\_mat} type. As in the pure QCD case, convolutions with
QED splitting functions can be represented by the \ttt{.conv.} operator or
using the product sign \ttt{*}.


\myparagraph{PDF arrays with photons and leptons}

A call to the subroutine
\begin{lstlisting}  
  subroutine AllocPDFWithPhoton(grid, pdf)
    type(grid_def), intent(in) :: grid
    real(dp),       pointer    :: pdf(:,:)
    [...]
  end subroutine AllocPDFWithPhoton
\end{lstlisting}
allocates PDFs (\ttt{pdf}) including photons, while a call to
\begin{lstlisting}  
  subroutine AllocPDFWithLeptons(grid, pdf)
    type(grid_def), intent(in) :: grid
    real(dp),       pointer    :: pdf(:,:)
    [...]
    end subroutine AllocPDFWithLeptons
\end{lstlisting}
allocates PDFs including both photons and leptons.
%
The two dimensions of the \ttt{pdf} refer to
the index of the $x$ value in the \ttt{grid}, and to the flavour index.
The flavour indices for photons and leptons %in the arrays \ttt{pdf}
are
given by
\begin{lstlisting}  
  integer, parameter, public :: iflv_photon   =  8
  integer, parameter, public :: iflv_electron =  9
  integer, parameter, public :: iflv_muon     = 10
  integer, parameter, public :: iflv_tau      = 11
\end{lstlisting}
where each \ttt{pdf(:,9:11)} contains the sum of a lepton and
anti-lepton flavour
(which are identical). Note that if one extends the
calculation of lepton PDFs to higher order in $\alpha$,
then an asymmetry in the lepton and anti-lepton distribution would arise,
due to the  \ttt{Plq\_03} splitting function\ak{I think this needs more explanation}.
% \footnote{
%   Were one to use \hoppet to obtain the distribution of
%   partons inside a lepton, then it would become useful to separate
%   lepton and anti-lepton PDFs.
%   % 
%   However further considerations arise, notably for the treatment of
%   the $1-x \ll 1$ region.\commentgz{explain better?} 
%   % 
%   See for example discussion in Ref.~\cite{Frixione:2023gmf}\commentgz{update refs}.}

The subroutine \ttt{AllocPDFWithPhotons} allocates the \ttt{pdf}
array with the flavour index up to 8,
while in the subroutine \ttt{AllocPDFWithLeptons},
the flavour index extends to 11.


\myparagraph{PDF tables with photons and leptons}

Next one needs to prepare a \ttt{pdf\_table} object forming the
interpolating grid for the evolved PDF's. We recall that the
\ttt{pdf\_table} object contains a table \ttt{pdf\_table\%tab(:,:,:)},
where the first index loops over $x$ values, the second loops over
flavours and the last loops over $Q^2$ values.


This is initialized by a call to 
\begin{lstlisting}
  subroutine AllocPdfTableWithLeptons(grid, pdftable, Qmin, Qmax & 
  & [, dlnlnQ] [, lnlnQ_order] [, freeze_at_Qmin])
   use qed_objects
   type(grid_def),    intent(in)    :: grid
   type(pdf_table),   intent(inout) :: pdftable
   real(dp), intent(in)             :: Qmin, Qmax
   real(dp), intent(in), optional   :: dlnlnQ
   integer,  intent(in), optional   :: lnlnQ_order
   logical,  intent(in), optional   :: freeze_at_Qmin
   [...]
   end subroutine AllocPdfTableWithLeptons
\end{lstlisting}
that is identical to the one without photon or leptons, the only difference is that the
maximum pdf flavour index in \ttt{pdf\_table\%tab} now includes the photon and leptons.
%
An analogous subroutine \ttt{AllocPdfTableWithPhoton} includes the
photon but no leptons.

\myparagraph{Evolution with photons and leptons}


To fill a table via an evolution from an initial scale, one calls the subroutine 
\begin{lstlisting}
    subroutine EvolvePdfTableQED(table, Q0, pdf0, dh, qed_split, &
    &  coupling, coupling_qed, nloop_qcd, nloopqcd_qed, with_Plq_nnloqed)
       type(pdf_table),        intent(inout) :: table
       real(dp),               intent(in)    :: Q0
       real(dp),               intent(in)    :: pdf0(:,:)
       type(dglap_holder),     intent(in)    :: dh
       type(qed_split_mat),    intent(in)    :: qed_split
       type(running_coupling), intent(in)    :: coupling
       type(qed_coupling),     intent(in)    :: coupling_qed
       integer,                intent(in)    :: nloop_qcd, nloopqcd_qed
       logical,  optional,     intent(in)    :: with_Plq_nnloqed
       [...]
    end subroutine EvolvePdfTableQED   
\end{lstlisting}
where \ttt{table} is the output, \ttt{Q0} the initial scale and
\ttt{pdf0} is the PDF at the initial scale.
%
We recall that the lower and upper limits on 
scales in the table are as set at initialisation time for the table.
%
When \ttt{nloopqcd\_qed} is set to 1 (0) mixed QCD-QED effects are
(are not) included in the evolution.
%
Setting the variable \ttt{with\_Plq\_nnloqed=.true.} includes also the NNLO $P_{lq}$
splittings in the evolution.

To perform the evolution \ttt{EvolvePdfTableQED} calls the routine 
\begin{lstlisting}
    subroutine QEDQCDEvolvePDF(dh, qed_sm, pdf, coupling_qcd, coupling_qed,&
    &                     Q_init, Q_end, nloop_qcd, nqcdloop_qed, with_Plq_nnloqed)
       type(dglap_holder),     intent(in), target :: dh
       type(qed_split_mat),    intent(in), target :: qed_sm
       type(running_coupling), intent(in), target :: coupling_qcd
       type(qed_coupling),     intent(in), target :: coupling_qed
       real(dp),               intent(inout)      :: pdf(0:,ncompmin:)
       real(dp),               intent(in)         :: Q_init, Q_end
       integer,                intent(in)         :: nloop_qcd
       integer,                intent(in)         :: nqcdloop_qed
       logical,  optional,     intent(in)         :: with_Plq_nnloqed
       [...]
       end subroutine QEDQCDEvolvePDF
\end{lstlisting}
which, given the \ttt{pdf} at an initial scale \ttt{Q\_init}, it
evolves it to scale \ttt{Q\_end}, overwriting the \ttt{pdf} array. 
%
In order to get interpolated PDF values from the table we use the
\ttt{EvalPdfTable\_*} calls, described in
\ifreleasenote
Section~7.2 of Ref.~\cite{Salam:2008qg}.
\else
Section~\ref{sec:acc_table}.
\fi
%
However the \texttt{pdf} array that is passed as an argument and that
is set by those subroutines should range not from \texttt{(-6:6)} but
instead from \texttt{(-6:8)} if the PDF just has photons and
\texttt{(-6:11)} if the PDF also includes leptons.\footnote{Index $7$
is a historical artefact associated with internal \hoppet bookkeeping
and should be ignored in the resulting \ttt{pdf} array.}

Note that at the moment, when QED effects are included, cached
evolution is not supported.


\subsection{Streamlined interface with QED effects}
The streamlined interface including QED effects works as in the case of pure QCD evolution.
One has to add the following call
\begin{lstlisting}
  logical use_qed, use_qcd_qed, use_Plq_nnlo
  ...
  call hoppetSetQED(use_qed, use_qcd_qed, use_Plq_nnlo)
\end{lstlisting}
before using the streamlined interface routines.

The \ttt{use\_qed} argument turns on/off QED evolution at order
${\cal O}(\alpha)$ (i.e.\ items \ref{item:qed1} and \ref{item:qed2} in
the enumerated list at the beginning of
Sec.~\ref{sec:qed-evolution}).
%
The \ttt{use\_qcd\_qed} one turns
on/off mixed QCD$\times$QED effects in the evolution (i.e.\ item \ref{item:qed3}) and \ttt{use\_Plq\_nnlo} turns on/off
the order $\alpha^2 P_{\ell q}$ splitting function (i.e.\
item \ref{item:qed4}).
%
Without this call, all QED corrections
are off.


%======================================================================
%======================================================================
\section{Hadronic Structure Functions}
\label{sec:structure-funcs}
As of \hoppet version 1.3.0 the code provides access to the hadronic
structure functions. The structure functions are expressed as
convolutions of a set of massless hard coefficient functions and PDFs,
and make use of the tabulated PDFs and streamlined interface as
described in the sections above. The structure functions are provided
such that they can be used directly for cross section computations in
DIS or VBF, the latter of which has already been implemented in the
{\tt proVBFH}
package~\cite{Cacciari:2015jma,Dreyer:2016oyx,Dreyer:2018qbw,Dreyer:2018rfu},
for which this work was originally developed.

The structure functions have been found to be in good agreement with
those which can be obtained with APFEL++~\cite{Bertone:2017gds} (at
the level of $10^{-4}$ relative precision).
%
The benchmarks with APFEL++ and the code used to carry
them out are described in detail in Ref.~\cite{bertonekarlberg} and
can be found in
\masterlink{benchmarking/structure\_functions\_benchmark\_checks.f90}.
%
Technical details on the implementation of the structure functions in
\hoppet can also be found therein, and in
Refs.~\cite{Dreyer:2016vbc,Karlberg:2016zik}. An example of their
usage can be found in
\masterlink{example\_f90/structure\_functions\_example.f90}.


The structure functions have been implemented up to
N$^3$LO\footnote{With the caveat that the 4-loop splitting functions
which are needed to claim this accuracy are not currently fully
known. At the time of writing partial results have been presented in
Refs.~\cite{Moch:2021qrk,Falcioni:2023luc,Falcioni:2023vqq,Gehrmann:2023cqm,Falcioni:2023tzp,Moch:2023tdj,Gehrmann:2023iah}. The
splitting functions up to three-loops were already implemented in
\hoppet as of version 1.1.0, and can be found
in~\cite{Furmanski:1980cm,Curci:1980uw,NNLO-NS,NNLO-singlet}} using
both the exact and parametrised coefficient functions found in
Refs.~\cite{vanNeerven:1999ca,vanNeerven:2000uj,Moch:2004xu,Vermaseren:2005qc,Moch:2008fj,Davies:2016ruz}.
The coefficient functions were also computed in
Ref.~\cite{Blumlein:2022gpp} and found to be in agreement with the
previous calculations.

\subsection{Initialisation}
\label{sec:structure-funcs-init}

The structure functions can be accessed from inside the module
\ttt{structure\_functions}. They can also be accessed through the
streamlined interface by prefixing \ttt{hoppet} as described in
section~\ref{sec:structure-functions-streamlined}.
%
The description below corresponds to a high-level interface, which
relies on elements such as the \ttt{grid} and splitting functions
having been initialised in the streamlined interface, through a call
to \ttt{hoppetStart} or \ttt{hoppetStartExtended}, cf.\
\ifreleasenote
Section~8 of Ref.~\cite{Salam:2008qg}.
\else
Section~\ref{sec:vanilla}.
\fi
\footnote{Users needing a lower level
  interface should inspect the code in \masterlink{src/structure\_functions.f90}.}
%
Then one calls
\begin{lstlisting}
  call StartStrFct(order_max [, nflav] [, xR] [, xF] [, scale_choice] &
                  & [, constant_mu] [ ,param_coefs] [ ,wmass] [ ,zmass])
\end{lstlisting}
specifying as a minimum the perturbative order --- currently
\ttt{order\_max}$\le 4$ (\ttt{order\_max}$=1$ corresponds to LO).

If \ttt{nflav} is not passed as an argument, the structure functions
are initialised to support a variable flavour-number scheme (the
masses that are used at any given stage will be those set in the
streamlined interface).
%
Otherwise a fixed number of light flavours is used, as indicated by
\ttt{nflav}, which speeds up initialisation.
%
Note that specifying a variable flavour-number scheme only has an
impact on the evolution and on $n_f$ terms in the coefficient
functions.
%
The latter, however always assume massless quarks.
%
Hence in a variable flavour-number scheme the structure functions
should only be considered reliable for $Q \gg m$.

Together \ttt{xR}, \ttt{xF}, \ttt{scale\_choice}, and \ttt{constant\_mu} control
the renormalisation and factorisation scales and the degree of
flexibility that will be available in choosing them at later stages.
%
Specifically \ttt{scale\_choice} can take several values:
\begin{itemize}
\item \ttt{scale\_choice\_Q} (default) means that the code will always
  use $Q$ multiplied by \ttt{xR} or \ttt{xF} as the renormalisation
  and factorisation scale respectively (with \ttt{xR} or \ttt{xF} as
  set at initialisation).
\item \ttt{scale\_choice\_fixed} corresponds to a fixed scale
  \ttt{constant\_mu} (multiplied by \ttt{xR} or \ttt{xF} as set at
  initialisation).
\item \ttt{scale\_choice\_arbitrary} allows the user to choose
  arbitrary scales at the moment of evaluating the structure
  functions.
  %
  In this last case, the structure functions are saved as arrays
  separated by perturbative order and with dedicated additional arrays
  for terms proportional to logarithms of $Q/\mu_F$. 
  % 
  This makes for a slower evaluation compared to the two other
  scale choices.
\end{itemize}
%
% The default value for \ttt{scale\_choice} is \ttt{scale\_choice\_Q} which means that $Q$
% multiplied by \ttt{xR} or \ttt{xF} is used as the renormalisation or
% factorisation scale respectively.
% %
% The choice \ttt{scale\_choice\_fixed} corresponds to a fixed scale
% \ttt{constant\_mu} (multiplied by \ttt{xR} or \ttt{xF}).
% %
% Should a user wish to use some arbitrary scale choice,
% \ttt{scale\_choice} should be set to \ttt{scale\_choice\_arbitrary}.
% %
% \gps{Do we want to introduce some constants such as
%   \ttt{scale\_choice\_Q=1}, \ttt{scale\_choice\_fixed=0}, \ttt{scale\_choice\_arbitrary=2}?} \ak{Yes, and done!}
% %
% In this last case, the structure functions are saved as arrays not only in
% $Q$ but also $\mu_R$ and $\mu_F$.
% %
% This makes for a slightly slower evaluation compared to the two other
% scale choices.

If \ttt{param\_coefs} is set to \ttt{.true.} (its default) then the
structure functions are computed using the NNLO and N3LO
parametrisations found in
Refs.~\cite{vanNeerven:1999ca,vanNeerven:2000uj,Moch:2004xu,Vermaseren:2005qc,Moch:2008fj,Davies:2016ruz},
which are stated to have a relative precision of a few per mil (order
by order) except at particularly small or large values of $x$.
%
Otherwise
the exact versions are used\footnote{The LO and NLO coefficient
functions are always exact as their expressions are very
compact.}.
%
This however means that the initialisation becomes slow (about two
minutes rather than a few seconds).
%
Given the good accuracy of the parametrised coefficient functions,
they are to be preferred for most applications.
% 
% and since the parametrised
% expressions are good to a relative accuracy of $10^{-4}$ for most
% values of $x$, it is recommended to use the parametrised option for
% most applications.
% 
Note that the exact expressions also add to compilation time and need
to be explicitly enabled with the \ttt{--enable-exact-coefs} configure
flag.\footnote{ At N3LO they rely on an extended version of
  \ttt{hplog}~\cite{FortranPolyLog}, \texttt{hplog5} version 1.0, that
  is able to handle harmonic polylogarithms up to weight 5.
  %
  However there appears to be a percent-level issue on the coefficient
  functions for $x\lesssim 0.1$, connected with the evaluation of the
  polylogarithms. This is currently under investigation. }

The mass of the electroweak vector bosons are used only to calculate
the weak mixing angle, $\sin^2 \theta_W = 1 - (m_W/m_Z)^2$.

At this point all the tables that are needed for the structure
functions have been allocated.
%
In order to fill the tables, one first needs to set up the running
coupling and evolve the initial PDF with \ttt{hoppetEvolve}, as
described in
\ifreleasenote
Section~8.2 of Ref.~\cite{Salam:2008qg}.
\else
Section~\ref{sec:vanilla_usage}.
\fi

%
% Care should be taken here such that both the coupling and PDF
% evolution are carried out at the correct perturbative order and with
% mass thresholds as appropriate.
%
With the PDF table filled in the streamlined interface one calls
\begin{lstlisting}
  call InitStrFct(order[, separate_orders])
\end{lstlisting}
specifying the order at which one would like to compute the structure
functions.
%
The flag \ttt{separate\_orders} should be set to \ttt{.true.}\ if one
wants access to the individual coefficients of the perturbative
expansion as well as the sum up to some maximum order.
%
With \ttt{scale\_choice\_Q} and \ttt{scale\_choice\_fixed}, the
default of \ttt{.false.}\ causes only the sum over perturbative orders
to be stored.
%
This gives faster evaluations of structure functions because it is
only necessary to interpolate for the sum over orders, rather than
interpolate one table for each order.
%
With \texttt{scale\_choice\_arbitrary}, the default is \ttt{.true.},
which is the only allowed option, because separate tables for each
order are required for the underlying calculations.

\subsection{Accessing the Structure Functions}
\label{sec:structure-funcs-access}
At this point the structure functions can be accessed as in the following example
\begin{lstlisting}
  real(dp) :: ff(-6:7), x, Q, muR, muF
  [...]
  call InitStrFct(order_max = 4)
  ff = StrFct(x, Q[, muR] [, muF])
\end{lstlisting}
at the value $x$ and $Q$.
%
With \ttt{scale\_choice\_arbitrary}, the \ttt{muR} and \ttt{muF}
arguments must be provided.
%
With other scale choices, they do not need to be provided, but if they
are then they should be consistent with the original scale choice.
%
The structure functions are in this example stored in the
array \ttt{ff}. The components of this array can be accessed through
the indices
\begin{lstlisting}
  integer, parameter :: iF1Wp= 1   
  integer, parameter :: iF2Wp= 2   
  integer, parameter :: iF3Wp= 3   
  integer, parameter :: iF1Wm=-1   
  integer, parameter :: iF2Wm=-2   
  integer, parameter :: iF3Wm=-3   
  integer, parameter :: iF1Z = 4   
  integer, parameter :: iF2Z = 5   
  integer, parameter :: iF3Z = 6   
  integer, parameter :: iF1EM = -4   
  integer, parameter :: iF2EM = -5   
  integer, parameter :: iF1gZ = 0  
  integer, parameter :: iF2gZ = -6 
  integer, parameter :: iF3gZ = 7  
\end{lstlisting}
For instance one would access the electromagnetic $F_1$ structure
function through \ttt{ff(iF1EM)}. It is returned at the \ttt{order\_max}
that was specified in \ttt{InitStrFct}.
%
The structure functions can also be accessed order by order if the
\ttt{separate\_orders} flag was set to \ttt{.true.} when initialising.
%
They are then obtained as follows
\begin{lstlisting}
  real(dp) :: flo(-6:7), fnlo(-6:7), fnnlo(-6:7), fn3lo(-6:7), x, Q, muR, muF
  [...]
  call InitStrFct(4, .true.)
  flo   = F_LO(x, Q, muR, muF)
  fnlo  = F_NLO(x, Q, muR, muF)
  fnnlo = F_NNLO(x, Q, muR, muF)
  fn3lo = F_N3LO(x, Q, muR, muF)
\end{lstlisting}
The functions return the individual contributions at each order in
$\as$, including the relevant factor of $\as^n$.
%
Hence the sum of \ttt{flo}, \ttt{fnlo}, \ttt{fnnlo}, and
\ttt{fn3lo} would return the full structure function at N3LO as
contained in \ttt{ff} in the example above.
%
Note that in the \ttt{F\_LO} etc.\ calls, the \ttt{muR} and \ttt{muF}
arguments are not optional and that when a prior scale choice has been
made (e.g. \ttt{scale\_choice\_Q}) they are assumed to be consistent
with that prior scale choice.

\subsection{Streamlined interface}
\label{sec:structure-functions-streamlined}
The structure functions can also be accessed through the streamlined
interface, so that they may be called for instance from C/C++. The
functions to be called are very similar to those described above. In
particular a user should call either
\begin{lstlisting}
  call hoppetStartStrFct(order_max)
\end{lstlisting}
with \ttt{order\_max} the maximal
order in $\as$ or, alternatively, the extended version of the interface
\ttt{hoppetStartStrFctExtended} which takes all the same arguments as
\ttt{StartStrFct} described above. One difference is that in order to
use a variable flavour scheme the user should set \ttt{nflav} to a
negative value. After evolving or reading in a PDF, the user then calls
\begin{lstlisting}
  call hoppetInitStrFct(order, separate_orders)
\end{lstlisting}
to initialise the actual structure functions. The structure functions
can then be accessed through the subroutines
\begin{lstlisting}
  real(dp) :: ff(-6:7), flo(-6:7), fnlo(-6:7), fnnlo(-6:7), fn3lo(-6:7), x, Q, muR, muF
  [...]
  call hoppetStrFct(x, Q, muR, muF, ff)        ! Full structure function
  ! or instead, without muR and muF
  call hoppetStrFctNoMu(x, Q, ff)              ! Full structure function
  call hoppetStrFctLO(x, Q, muR, muF, flo)     ! LO term
  call hoppetStrFctNLO(x, Q, muR, muF, fnlo)   ! NLO term
  call hoppetStrFctNNLO(x, Q, muR, muF, fnnlo) ! NNLO term
  call hoppetStrFctN3LO(x, Q, muR, muF, fn3lo) ! N3LO term
\end{lstlisting}
The C++ header contains indices for the structure functions and scale
choices, which are all in the \ttt{hoppet} namespace.
%
\begin{lstlisting}
  int iF1Wp = 1+6;
  int iF2Wp = 2+6;
  int iF3Wp = 3+6;
  int iF1Wm =-1+6;
  int iF2Wm =-2+6;
  int iF3Wm =-3+6;
  int iF1Z  = 4+6;
  int iF2Z  = 5+6;
  int iF3Z  = 6+6;
  int iF1EM =-4+6;
  int iF2EM =-5+6;
  int iF1gZ = 0+6;
  int iF2gZ =-6+6;
  int iF3gZ = 7+6;

  const int scale_choice_fixed     = 0;
  const int scale_choice_Q         = 1;
  const int scale_choice_arbitrary = 2;
\end{lstlisting}
Note that the structure function indices start from 0 and that the C++
array that is to be passed to functions such as \ttt{hoppetStrFct}
would be defined as \ttt{double ff[14]}.

%======================================================================
%======================================================================


%======================================================================
\section{Conclusion}

The \hoppet additions described here provide fast access to QED
evolution and to structure-function calculations, including evolution
in the VFNS at N$^3$LO, which we hope may be of benefit beyond the
projects where they have already been used.


\section*{Acknowledgments}

\comment{Add Melissa}

We gratefully acknowledge the work of Luca Buonocore
for the implementation of the $P_{lq}$ splitting function in the QED code.
%
We are also grateful to Johannes Bl\"umlein for providing us with a
pre-release version of the code in Ref.~\cite{BlumleinCode} and a
suitable license for its use.
%
We also wish to thank Juan Rojo for useful discussions. 
%
GPS wishes to acknowledge funding from a Royal Society Research
Professorship (grant RP$\backslash$R$\backslash$231001) and from the Science and
Technology Facilities Council (STFC) under grant ST/X000761/1.
%
PN thanks the Humboldt Foundation for support. 
%
\appendix

\section{Perturbative evolution in QCD}
\label{sec:pqcd}
First of all we set up the notation and
conventions that are used throughout \hoppet. The DGLAP
equation for a non-singlet parton distribution reads
\begin{equation}
  \label{eq:dglap-ns}
  \frac{\partial q(x,Q^2)}{\partial \ln Q^2} = 
\frac{\aq}{2\pi}\int_x^1 \frac{dz}{z}
  P(z,\aq) q\lp \frac{x}{z},Q^2\rp \equiv 
\frac{\aq}{2\pi}  P(x,\aq) \otimes q\lp x,Q^2\rp \ .
\end{equation}
The related variable $t\equiv \ln Q^2$ is also used
in various places in \hoppet.
The splitting functions in eq.~(\ref{eq:dglap-ns})
are known exactly up to NNLO in the 
unpolarised case \cite{Furmanski:1980cm,Curci:1980uw,NNLO-NS,NNLO-singlet}, and approximately at N$^3$LO~\cite{Gracey:1994nn,Davies:2016jie,Moch:2017uml,Gehrmann:2023cqm,Falcioni:2023tzp,Gehrmann:2023iah,McGowan:2022nag,NNPDF:2024nan,Moch:2021qrk,Falcioni:2023luc,Falcioni:2023vqq,Moch:2023tdj,Falcioni:2024xyt,Falcioni:2024qpd}:
\begin{equation}
  \label{eq:dpdf}
   P(z,\aq)=P^{(0)}(z)+\frac{\aq}{2\pi}P^{(1)}(z)+
\lp \frac{\aq}{2\pi} \rp^2 P^{(2)}(z) 
+
\lp \frac{\aq}{2\pi} \rp^3 P^{(3)}(z) \ ,
\end{equation}
and up to NNLO \cite{Mertig:1995ny,Vogelsang:1996im,Moch:2014sna,Moch:2015usa,Blumlein:2021enk,Blumlein:2021ryt} in the polarised case.
The generalisation to the singlet case is straightforward, as it
is 
%the generalisation of eq.~(\ref{eq:dglap-ns}) 
to the case of time-like evolution\footnote{
The general structure of the relation between space-like
and time-like evolution and splitting functions
 has been investigated in \cite{Furmanski:1980cm,Curci:1980uw,Stratmann:1996hn,Dokshitzer:2005bf,Mitov:2006ic,Basso:2006nk,Dokshitzer:2006nm,Beccaria:2007bb}.\ak{This list may also need some update..}}, 
relevant for example for fragmentation function analysis,
where NNLO results
are also available \cite{Mitov:2006ic,Moch:2007tx,Almasy:2011eq}.


As with the splitting functions, all perturbative quantities in
\hoppet are defined to be coefficients of powers of $\as/2\pi$. The one
exception is the $\beta$-function coefficients of the running
coupling equation:
\begin{equation}
  \label{eq:as-ev}
  \frac{d\as}{d\ln Q^2} = \beta\lp \aq\rp = -\as (\beta_0\as +
  \beta_1\as^2 + 
  \beta_2\as^3 + 
  \beta_3\as^4) \ .
\end{equation}

The evolution of the strong coupling and the parton distributions can
be performed in both the fixed flavour-number scheme (FFNS) and the 
variable flavour-number scheme (VFNS). In the VFNS case we 
need the matching conditions between the effective
theories with $n_f$ and $n_{f}+1$ light flavours for both the strong 
coupling $\aq$ and the parton distributions at the heavy quark
mass threshold $m_h^2$.

These matching conditions for the parton distributions
receive non-trivial contributions at higher orders. In the $\MSbar$
(factorisation) scheme, for example,
carrying out the matching at a scale equal to the heavy-quark mass
these begin at NNLO:\footnote{In
  a general scheme they would start at NLO.} %
for light quarks $q_{l,i}$ of flavour $i$ 
(quarks that are considered massless
below the heavy quark mass threshold $m_h^2$) the matching between
their values in the $n_f$ and
$n_f+1$ effective theories reads\ak{Do we need a footnote pointing out that this structure is not entirely clear from the literature?}:
%\begin{equation}
%\label{eq:lp-nf1}
%  q_{l,i}^{\,(n_f+1)}(x,m_h^2) \: = \:  q_{l,i}^{\,(\nf}(x,m_h^2) +
%\lp \frac{\alpha_s(m_h^2)}{2\pi} \rp^2
%   A^{\rm ns,(2)}_{qq,h}(x) \otimes
%  q_{l,i}^{\, (\nf}(x,m_h^2) \ ,
%\end{equation}
\begin{align}
\label{eq:lp-nf1}
  q_{l,i}^{\,(n_f+1)}(x,m_h^2) + q_{l,-i}^{\,(n_f+1)}(x,m_h^2)  & =   q_{l,i}^{\,(\nf}(x,m_h^2) + q_{l,-i}^{\,(\nf}(x,m_h^2) \notag \\ &+
   A^{\rm NS,+}_{qq,h}(x) \otimes \left(
   q_{l,i}^{\, (\nf}(x,m_h^2) + q_{l,-i}^{\, (\nf}(x,m_h^2)\right) \notag\\
   & + \frac{1}{n_f} \Big\{A^{\rm PS}_{qq,h}(x) \otimes \Sigma^{\, (\nf}(x,m_h^2) \notag\\
   & + A^{\rm S}_{qg,h}(x) \otimes g^{\, (\nf}(x,m_h^2)\Big\} \ , \notag \\
  q_{l,i}^{\,(n_f+1)}(x,m_h^2) - q_{l,-i}^{\,(n_f+1)}(x,m_h^2)  & =   q_{l,i}^{\,(\nf}(x,m_h^2) - q_{l,-i}^{\,(\nf}(x,m_h^2) \notag \\ &+
   A^{\rm NS,-}_{qq,h}(x) \otimes \left(
   q_{l,i}^{\, (\nf}(x,m_h^2) - q_{l,-i}^{\, (\nf}(x,m_h^2)\right) \ ,
\end{align}
where $i = 1,\ldots n_f$, while for the gluon distribution, the heavy
quark PDF $q_h$, and the singlet PDF $\Sigma(x,Q^2)$ (defined in
Table~\ref{eq:diag_split}) one has :
\begin{align}
\label{eq:hp-nf1}
  g^{(n_f+1)}(x,m_h^2)  &=
    g^{\, (\nf}(x,m_h^2) +
    A_{\rm gq,h}^{\rm S}(x) \otimes \Sigma^{(\nf}(x,m_h^2) +
    A_{\rm gg,h}^{\rm S}(x) \otimes g^{(\nf}(x,m_h^2) \ ,
  \nn \\[0.3cm]
  (q_h+\bar{q}_{h})^{(n_f+1)}(x,m_h^2)  &=
  A_{\rm hq}^{\rm S}(x)\otimes \Sigma^{(\nf}(x,m_h^2) 
  + A_{\rm hg}^{\rm S}(x)\otimes g^{(\nf}(x,m_h^2)\ ,  \nn \\[0.3cm]
  \Sigma^{(n_f+1)}(x,m_h^2)  &= \Sigma^{\, (\nf}(x,m_h^2) + \left[ A^{\rm NS,+}_{qq,h}(x) + A^{\rm PS}_{qq,h}(x) + A_{\rm hq}^{\rm S}(x)\right] \otimes \Sigma^{(\nf}(x,m_h^2) \nn \\
  & + \left[ A^{\rm S}_{qg,h}(x) + A_{\rm hg}^{\rm S}(x) \right] \otimes g^{(\nf}(x,m_h^2)
\end{align}
with $q_h=\bar{q}_h$. Up to N$^3$LO the matching coefficients have the
following expansions in $\alpha_s$
\begin{align}
  A^{\rm NS,\pm}_{qq,h}(x) & = \lp \frac{\alpha_s(m_h^2)}{2\pi} \rp^2
  A^{\rm NS,\pm,(2)}_{qq,h}(x) + \lp \frac{\alpha_s(m_h^2)}{2\pi} \rp^3
  A^{\rm NS,\pm,(3)}_{qq,h}(x) \ , \nn \\
  A^{\rm S}_{gk,h}(x) & = \lp \frac{\alpha_s(m_h^2)}{2\pi} \rp^2
  A^{\rm S,(2)}_{gk,h}(x) + \lp \frac{\alpha_s(m_h^2)}{2\pi} \rp^3
  A^{\rm S,(3)}_{gk,h}(x), \quad k=q,g \ , \nn \\
  %A^{\rm S}_{gg,h}(x) & = \lp \frac{\alpha_s(m_h^2)}{2\pi} \rp^2
  %A^{\rm S,(2)}_{gg,h}(x) + \lp \frac{\alpha_s(m_h^2)}{2\pi} \rp^3
  %A^{\rm S,(3)}_{gg,h}(x) \ , \nn \\
  A^{\rm S}_{hk}(x) & = \lp \frac{\alpha_s(m_h^2)}{2\pi} \rp^2
  A^{\rm S,(2)}_{hk}(x) + \lp \frac{\alpha_s(m_h^2)}{2\pi} \rp^3
  A^{\rm S,(3)}_{hk}(x), \quad k=q,g \ , \nn \\
  %A^{\rm S}_{hg}(x) & = \lp \frac{\alpha_s(m_h^2)}{2\pi} \rp^2
  %A^{\rm S,(2)}_{hg}(x) + \lp \frac{\alpha_s(m_h^2)}{2\pi} \rp^3
  %A^{\rm S,(3)}_{hg}(x) \ , \nn \\
  A^{\rm PS}_{qq,h}(x) & = \lp \frac{\alpha_s(m_h^2)}{2\pi} \rp^3
  A^{\rm PS,(3)}_{qq,h}(x) \ , \nn \\
  A^{\rm S}_{qg,h}(x) & = \lp \frac{\alpha_s(m_h^2)}{2\pi} \rp^3
  A^{\rm S,(3)}_{qg,h}(x)
\end{align}
At $\mathcal{O}(\alpha_S^2)$ we have that $A^{\rm NS,+}_{qq,h}(x) =
A^{\rm NS,-}_{qq,h}(x)$ whereas they start to differ at
$\mathcal{O}(\alpha_S^3)$. The NNLO matching coefficients were
computed in \cite{NNLO-MTM}\footnote{The authors are thanked for the
code corresponding to the calculation.} and the N$^3$LO matching
coefficients
in~\cite{Bierenbaum:2009mv,Ablinger:2010ty,Kawamura:2012cr,Blumlein:2012vq,ABLINGER2014263,Ablinger:2014nga,Ablinger:2014vwa,Behring:2014eya,Ablinger:2019etw,Behring:2021asx,Fael:2022miw,Ablinger:2023ahe,Ablinger:2024xtt,BlumleinCode}\footnote{We
thank Johannes Bl\"umlein for sharing the code in
Ref.~\cite{BlumleinCode} with us, which also contains code associated
with Refs.~\cite{Ablinger:2024xtt,Fael:2022miw}.
%
}
%
Notice that the above
conditions will lead to small discontinuities of the PDFs in its
evolution in $Q^2$, which are cancelled by similar matching terms in
the coefficient functions in VFN schemes resulting in continuous physical
observables. In particular, the heavy quark PDFs start from a non-zero
value at threshold at NNLO, which sometimes can even be negative.

The corresponding NNLO relation for the matching of the $\MSbar$
coupling constant at the heavy quark threshold $m^2_h$ is given by 
\begin{equation}
\label{eq:as-nf1}
  \as^{\, (n_f+1)}(m_h^2) \: = \:
  \as^{\, (\nf} (m_h^2) +   C_2 \lp \frac{\as^{\, (\nf} (m_h^2)}{2\pi} \rp^3+   C_3 \lp \frac{\as^{\, (\nf} (m_h^2)}{2\pi} \rp^4
   \:\: ,
\end{equation}
where the matching coefficients $C_2$ and $C_3$ were computed in
\cite{Chetyrkin:1997sg,Chetyrkin:1997un}.
%
The value and the form of the matching coefficients in
eqs.~(\ref{eq:lp-nf1},\ref{eq:hp-nf1}) depend on the scheme used for
the quark masses; by default in \hoppet quark masses are taken to be
pole masses, though the option exists for the user to supply and have
thresholds crossed at $\MSbar$ masses, but only up to NNLO. We note
that in the current implementation in \hoppet the matching can only be
performed at the matching point that corresponds to the quark masses
themselves.

Both evolution and threshold matching preserve the momentum sum rule
\begin{equation}
  \int_0^1 dx~x \lp \Sigma(x,Q^2)+g(x,Q^2)\rp =1 \,,
\end{equation}
and valence sum rules
\begin{equation}
  \int_0^1 dx\, \left[q(x,Q^2)-{\bar q}(x,Q^2) \right] = \left\{ 
    \begin{array}{ll}
      1, & \text{for } q = d \text{ (in proton)}\\
      2, & \text{for } q = u \text{ (in proton)}\\
      0, & \text{other flavours}
    \end{array}
    \right.
\end{equation}
as long as they hold at the initial scale (occasionally not the case,
\eg in modified LO sets for Monte Carlo
generators~\cite{Sherstnev:2008dm}).

The default basis for the PDFs, called the \ttt{human} 
representation in \hoppet, is such that 
 the entries in an array
\ttt{pdf(-6:6)} of PDFs correspond to:
\bea 
\bar t={-6} \ ,  \bar b={-5} \ ,  \bar c={-4}
\ , \nn   \bar s&=&{-3} \ , \nn  \bar u={-2} \ , \nn
 \bar d={-1} \ , \\  g&=&{0} \ , \\ \nn   d={1} \ , \nn  u={2} 
\ , \nn  
s={3} \ , \nn   c&=&{4} \ , \nn b={5} \ , \nn  t={6} \ . \nn 
\eea
 This representation is the
same as that used in the \ttt{LHAPDF} library \cite{LHAPDF}. 
However, this representation leads
to a complicated form of the evolution equations.
The splitting matrix can be simplified considerably (made diagonal
except for a $2\times2$ singlet block) by switching to a different
flavour representation, which is named
the \ttt{evln} representation, for the PDF set, as explained in detail in
\cite{vanNeerven:1999ca,vanNeerven:2000uj}. This representation
is described in Table \ref{eq:diag_split}.

In the {\tt evln} basis, 
the gluon evolves coupled to the singlet  PDF $\Sigma$,
and all non-singlet PDFs evolve independently.
Notice that the representations of the PDFs
are preserved under linear operations, so in particular
they are preserved under DGLAP evolution.
The conversion from the \ttt{human} to the \ttt{evln}
representations of PDFs requires that the number of
active quark flavours $n_f$ be specified by the user, as described in
\ifreleasenote
Section~5.1.2 of Ref.~\cite{Salam:2008qg}.
\else
Section~\ref{sec:evln-rep}.
\fi

\begin{table}
\begin{center}
\begin{tabular}{|r | c | l |}
\hline
     i & \mbox{name} & $q_i$ \\ \hline
     $ -6\ldots-(n_f+1)$ & $q_i$ & $q_i$\\
     $-n_f\ldots -2$ & $q_{\mathrm{NS},i}^{-}$ & 
$(q_i -  {\bar q}_i) - (q_1 - {\bar q}_1)$\\
      -1           & $q_{\mathrm{NS}}^{V}$ & 
$\sum_{j=1}^{n_f} (q_j -  {\bar q}_j)$\\
       0           & g & \textrm{gluon} \\
       1           & $\Sigma$ & $\sum_{j=1}^{n_f} (q_j +  {\bar q}_j)$\\
     $2\ldots n_f$ & $q_{\mathrm{NS},i}^{+}$ &
$ (q_i +  {\bar q}_i) - (q_1 + {\bar q}_1)$\\
      $(n_f+1)\ldots6$ & $q_i$ & $q_i$ \\
\hline
\end{tabular}
\caption{}{\label{eq:diag_split} The evolution representation 
(called \ttt{evln} in \hoppet)
of PDFs with $n_f$ active quark flavours
in terms of the \ttt{human} representation.}  
\end{center}
\end{table}

In \hoppet unpolarised DGLAP evolution is available up to N$^3$LO
in the $\MSbar$ scheme, while for the DIS scheme
only evolution up to NLO is available, but without the NLO heavy-quark
threshold matching conditions. For polarised evolution up to NLO only
the $\MSbar$ scheme is available. The variable \ttt{factscheme}
takes different values for each factorisation scheme:
\begin{center}
  \begin{tabular}{|c|l|}\hline
    \ttt{factscheme} & Evolution\\[2pt]\hline
    1 & unpolarised $\MSbar$ scheme\\[2pt]\hline
    2 & unpolarised DIS scheme\\[2pt]\hline
    3 & polarised $\MSbar$ scheme\\\hline
  \end{tabular}
\end{center}
Note that mass thresholds are currently
missing in the DIS scheme.


\bibliographystyle{elsarticle-num}
\bibliography{hoppet.bib}
%\begin{thebibliography}{99}

\bibitem{Botje}
  M.~Botje, QCDNUM, \url{http://www.nikhef.nl/~h24/qcdnum/}~.

% Uses decomposition on Laguerre polynomials -- about
% 30 of them, remains Y^2 * T method. Initialisation
% (transform of splitting functions takes 15s on thalie)
% (didn't try evolution; didn't check accuracy; evolution
% times and accuracy are not mentioned; seemed fixed nf)
\bibitem{Schoeffel:1998tz}
L.~Schoeffel,
%``An elegant and fast method to solve QCD evolution equations,  application to
%the determination of the gluon content of the pomeron,''
Nucl.\ Instrum.\ Meth.\ A {\bf 423} (1999) 439.
%%CITATION = NUIMA,A423,439;%%
See also \url{http://www.desy.de/~schoffel/L_qcd98.html},
\url{http://www-spht.cea.fr/pisp/gelis/Soft/DGLAP/index.html}

\bibitem{Pegasus}
  A.~Vogt,
  %``Efficient evolution of unpolarized and polarized parton distributions  with
  %QCD-PEGASUS,''
  Comput.\ Phys.\ Commun.\  {\bf 170} (2005) 65
  [arXiv:hep-ph/0408244].
  %%CITATION = HEP-PH 0408244;%%

\bibitem{Pascaud:2001bi}
C.~Pascaud and F.~Zomer,
%``A fast and precise method to solve the Altarelli-Parisi equations in x
%space,''
arXiv:hep-ph/0104013.
%%CITATION = HEP-PH 0104013;%%


\bibitem{Weinzierl:2002mv}
S.~Weinzierl,
%``Fast evolution of parton distributions,''
Comput.\ Phys.\ Commun.\  {\bf 148} (2002) 314
[arXiv:hep-ph/0203112];
%%CITATION = HEP-PH 0203112;%%
%\bibitem{Roth:2004ti}
M.~Roth and S.~Weinzierl,
%``QED corrections to the evolution of parton distributions,''
Phys.\ Lett.\ B {\bf 590} (2004) 190
[arXiv:hep-ph/0403200].
%%CITATION = HEP-PH 0403200;%%

% about 1 minute at NLO.
\bibitem{coriano} A.~Cafarella and C.~Coriano,
%``Direct solution of renormalization group equations of QCD in x-space: NLO
%implementations at leading twist,''
Comput.\ Phys.\ Commun.\  {\bf 160} (2004) 213
[arXiv:hep-ph/0311313];
%%CITATION = HEP-PH 0311313;%%
%\bibitem{Cafarella:2005zj}
  A.~Cafarella, C.~Coriano' and M.~Guzzi,
  %``NNLO logarithmic expansions and exact solutions of the DGLAP equations from
  %x-space: New algorithms for precision studies at the LHC,''
  Nucl.\ Phys.\  B {\bf 748} (2006) 253
  [arXiv:hep-ph/0512358];
  %%CITATION = NUPHA,B748,253;%%
  A.~Cafarella, C.~Coriano and M.~Guzzi,
  %``Precision Studies of the NNLO DGLAP Evolution at the LHC with CANDIA,''
  arXiv:0803.0462 [hep-ph].

\bibitem{GuzziThesis}
  M.~Guzzi, Ph.D. Thesis, Lecce University, 2006 [hep-ph/0612355].

\bibitem{nnpdf}
  L.~Del Debbio, S.~Forte, J.~I.~Latorre, A.~Piccione and J.~Rojo  [NNPDF
                  Collaboration],
  %``Neural network determination of parton distributions: The nonsinglet
  %case,''
  JHEP {\bf 0703} (2007) 039
  [arXiv:hep-ph/0701127].

\bibitem{Kosower:1997hg}
  D.~A.~Kosower,
  %``Evolution of parton distributions,''
  Nucl.\ Phys.\  B {\bf 506} (1997) 439
  [arXiv:hep-ph/9706213].

\bibitem{Ratcliffe:2000kp}
  P.~G.~Ratcliffe,
  %``A matrix approach to numerical solution of the DGLAP evolution
  %equations,''
  Phys.\ Rev.\  D {\bf 63}, 116004 (2001)
  [arXiv:hep-ph/0012376].
  %%CITATION = PHRVA,D63,116004;%%

\bibitem{DGLAP}
V.N.~Gribov and L.N.~Lipatov, 
%\sjnp{15}{1972}{438};
Sov.\ J.\ Nucl.\ Phys. {\bf 15} (1972) 438;
%``Deep Inelastic E P Scattering In Perturbation Theory,''
%[Sov.\ J.\ Nucl.\ Phys.\  {\bf 15} (1972) 438].
%%CITATION = YAFIA,15,781;%%
G.~Altarelli and G.~Parisi, 
%\npb{126}{1977}{298};
Nucl.\ Phys.\ B {\bf 126} (1977) 298;
%``Asymptotic Freedom In Parton Language,''
%%CITATION = NUPHA,B126,298;%%
Yu.L.~Dokshitzer, 
%\jetp{46}{1977}{641}.
Sov.\ Phys.\ JETP {\bf 46} (1977) 641.
%``Calculation Of The Structure Functions For Deep Inelastic Scattering And E+ E- Annihilation By Perturbation Theory In Quantum Chromodynamics.
%[Zh.\ Eksp.\ Teor.\ Fiz.\  {\bf 73} (1977) 1216].
%%CITATION = SPHJA,46,641;%%

\bibitem{CTEQ}
  J.~Pumplin, D.~R.~Stump, J.~Huston, H.~L.~Lai, P.~Nadolsky and W.~K.~Tung,
  %``New generation of parton distributions with uncertainties from global  QCD
  %analysis,''
  JHEP {\bf 0207}, 012 (2002)
  [arXiv:hep-ph/0201195].

%\cite{Martin:2002dr}
\bibitem{MRST}
  A.~D.~Martin, R.~G.~Roberts, W.~J.~Stirling and R.~S.~Thorne,
  %``NNLO global parton analysis,''
  Phys.\ Lett.\  B {\bf 531} (2002) 216
  [arXiv:hep-ph/0201127].
  %%CITATION = PHLTA,B531,216;%%

\bibitem{DisResum}
  M.~Dasgupta and G.~P.~Salam,
  %``Resummation of the jet broadening in DIS,''
  Eur.\ Phys.\ J.\  C {\bf 24} (2002) 213
  [arXiv:hep-ph/0110213];
  %%CITATION = EPHJA,C24,213;%%
%\bibitem{Dasgupta:2002dc}
  %M.~Dasgupta and G.~P.~Salam,
  %``Resummed event-shape variables in DIS,''
  JHEP {\bf 0208} (2002) 032
  [arXiv:hep-ph/0208073].
  %%CITATION = JHEPA,0208,032;%%



\bibitem{caesar}
  A.~Banfi, G.~P.~Salam and G.~Zanderighi,
  %``Principles of general final-state resummation and automated
  %implementation,''
  JHEP {\bf 0503}, 073 (2005)
  [arXiv:hep-ph/0407286];
  %%CITATION = JHEPA,0503,073;%%
%
%\cite{Banfi:2004nk}
%\bibitem{Banfi:2004nk}
%  A.~Banfi, G.~P.~Salam and G.~Zanderighi,
  %``Resummed event shapes at hadron hadron colliders,''
  JHEP {\bf 0408}, 062 (2004)
  [arXiv:hep-ph/0407287].
  %%CITATION = JHEPA,0408,062;%%

\bibitem{Ciafaloni:2003rd}
  M.~Ciafaloni, D.~Colferai, G.~P.~Salam and A.~M.~Stasto,
  %``Renormalisation group improved small-x Green's function,''
  Phys.\ Rev.\  D {\bf 68}, 114003 (2003)
  [arXiv:hep-ph/0307188].


\bibitem{APPL}
  T.~Carli, G.~P.~Salam and F.~Siegert,
  %``A posteriori inclusion of PDFs in NLO QCD final-state calculations,''
  :hep-ph/0510324;
  %%CITATION = HEP-PH/0510324;%%
  T.~Carli, D.~Clements, {\it et al.}, in preparation.

\bibitem{Banfi:2007gu}
  A.~Banfi, G.~P.~Salam and G.~Zanderighi,
  %``Accurate QCD predictions for heavy-quark jets at the Tevatron and LHC,''
  JHEP {\bf 0707} (2007) 026
  [arXiv:0704.2999 [hep-ph]].


\bibitem{Benchmarks} 
  W.~Giele {\it et al.},
  ``Les Houches 2001, the QCD/SM working group: Summary report,''
  hep-ph/0204316, section 1.3;\\
  %%CITATION = HEP-PH 0204316;%%
  M.~Dittmar {\it et al.},
  ``Parton distributions: Summary report for the HERA-LHC workshop,''
  hep-ph/0511119, section 4.4.
  %%CITATION = HEP-PH 0511119;%%

\bibitem{LHAPDF} W.~Giele and M.~R.~Whalley,
\url{http://hepforge.cedar.ac.uk/lhapdf/}


\bibitem{CFP}
  W.~Furmanski and R.~Petronzio,
  %``Singlet Parton Densities Beyond Leading Order,''
  Phys.\ Lett.\  B {\bf 97} (1980) 437;
  %%CITATION = PHLTA,B97,437;%%
  G.~Curci, W.~Furmanski and R.~Petronzio,
  %``Evolution Of Parton Densities Beyond Leading Order: The Nonsinglet Case,''
  Nucl.\ Phys.\  B {\bf 175} (1980) 27.
  %%CITATION = NUPHA,B175,27;%%


%\cite{Moch:2004pa}
\bibitem{NNLO-NS}
  S.~Moch, J.~A.~M.~Vermaseren and A.~Vogt,
  %``The three-loop splitting functions in QCD: The non-singlet case,''
  Nucl.\ Phys.\ B {\bf 688} (2004) 101
  [arXiv:hep-ph/0403192].
  %%CITATION = HEP-PH 0403192;%%

%\cite{Vogt:2004mw}
\bibitem{NNLO-singlet}
  A.~Vogt, S.~Moch and J.~A.~M.~Vermaseren,
  %``The three-loop splitting functions in QCD: The singlet case,''
  Nucl.\ Phys.\ B {\bf 691} (2004) 129
  [arXiv:hep-ph/0404111].
  %%CITATION = HEP-PH 0404111;%%

\bibitem{Mertig:1995ny}
  R.~Mertig and W.~L.~van Neerven,
  %``The Calculation Of The Two Loop Spin Splitting Functions P(Ij)(1)(X),''
  Z.\ Phys.\  C {\bf 70} (1996) 637
  [arXiv:hep-ph/9506451].
  %%CITATION = ZEPYA,C70,637;%%

\bibitem{Vogelsang:1996im}
  W.~Vogelsang,
  %``The spin-dependent two-loop splitting functions,''
  Nucl.\ Phys.\  B {\bf 475} (1996) 47
  [arXiv:hep-ph/9603366].
  %%CITATION = NUPHA,B475,47;%%


\bibitem{Stratmann:1996hn}
  M.~Stratmann and W.~Vogelsang,
  %``Next-to-leading order evolution of polarized and unpolarized  fragmentation
  %functions,''
  Nucl.\ Phys.\  B {\bf 496} (1997) 41
  [arXiv:hep-ph/9612250].
  %%CITATION = NUPHA,B496,41;%%

%\cite{Dokshitzer:2005bf}
\bibitem{Dokshitzer:2005bf}
  Yu.~L.~Dokshitzer, G.~Marchesini and G.~P.~Salam,
  %``Revisiting parton evolution and the large-x limit,''
  Phys.\ Lett.\  B {\bf 634}, 504 (2006)
  [arXiv:hep-ph/0511302].
  %%CITATION = PHLTA,B634,504;%%


\bibitem{Mitov:2006ic}
  A.~Mitov, S.~Moch and A.~Vogt,
  %``Next-to-next-to-leading order evolution of non-singlet fragmentation
  %functions,''
  Phys.\ Lett.\  B {\bf 638} (2006) 61
  [arXiv:hep-ph/0604053].

\bibitem{Basso:2006nk}
  B.~Basso and G.~P.~Korchemsky,
  %``Anomalous dimensions of high-spin operators beyond the leading order,''
  Nucl.\ Phys.\  B {\bf 775} (2007) 1
  [arXiv:hep-th/0612247].
  %%CITATION = NUPHA,B775,1;%%

\bibitem{Dokshitzer:2006nm}
  Yu.~L.~Dokshitzer and G.~Marchesini,
  %``N = 4 SUSY Yang-Mills: Three loops made simple(r),''
  Phys.\ Lett.\  B {\bf 646} (2007) 189
  [arXiv:hep-th/0612248].

\bibitem{Beccaria:2007bb}
  M.~Beccaria, Yu.~L.~Dokshitzer and G.~Marchesini,
  %``Twist 3 of the sl(2) sector of N=4 SYM and reciprocity respecting
  %evolution,''
  Phys.\ Lett.\  B {\bf 652} (2007) 194
  [arXiv:0705.2639 [hep-th]].



\bibitem{NNLO-MTM}
  M.~Buza, Y.~Matiounine, J.~Smith, R.~Migneron and W.~L.~van Neerven,
  %``Heavy quark coefficient functions at asymptotic values $Q~2 \gg m~2$,''
  Nucl.\ Phys.\ B {\bf 472}, 611 (1996)
  [arXiv:hep-ph/9601302];\\
  %%CITATION = HEP-PH 9601302;%%
%
  M.~Buza, Y.~Matiounine, J.~Smith and W.~L.~van Neerven,
  %``Charm electroproduction viewed in the variable-flavour number scheme
  %versus fixed-order perturbation theory,''
  Eur.\ Phys.\ J.\ C {\bf 1}, 301 (1998)
  [arXiv:hep-ph/9612398].
  %%CITATION = HEP-PH 9612398;%%

\bibitem{Chetyrkin:1997sg}
  K.~G.~Chetyrkin, B.~A.~Kniehl and M.~Steinhauser,
  %``Strong coupling constant with flavour thresholds at four loops in the
  %MS-bar scheme,''
  Phys.\ Rev.\ Lett.\  {\bf 79}, 2184 (1997)
  [arXiv:hep-ph/9706430].

%\cite{Sherstnev:2008dm}
\bibitem{Sherstnev:2008dm}
  A.~Sherstnev and R.~S.~Thorne,
  %``Different PDF approximations useful for LO Monte Carlo generators,''
  arXiv:0807.2132 [hep-ph].
  %%CITATION = ARXIV:0807.2132;%%

\bibitem{vanNeerven:1999ca}
  W.~L.~van Neerven and A.~Vogt,
  %``NNLO evolution of deep-inelastic structure functions: The non-singlet
  %case,''
  Nucl.\ Phys.\ B {\bf 568} (2000) 263
  [arXiv:hep-ph/9907472].
  %%CITATION = HEP-PH 9907472;%%

\bibitem{vanNeerven:2000uj}
  W.~L.~van Neerven and A.~Vogt,
  %``NNLO evolution of deep-inelastic structure functions: The singlet case,''
  Nucl.\ Phys.\ B {\bf 588} (2000) 345
  [arXiv:hep-ph/0006154].
  %%CITATION = HEP-PH 0006154;%%

\bibitem{NRf90}
  Press {\it et al.}, \emph{Numerical Recipes in Fortran~90},
  Cambridge University Press, 1996.
  
\bibitem{VogtMTMParam} A.~Vogt, private communication.


\bibitem{White:2005wm}
  C.~D.~White and R.~S.~Thorne,
  %``Comparison of NNLO DIS scheme splitting functions with results from exact
  %gluon kinematics at small x,''
  Eur.\ Phys.\ J.\ C {\bf 45} (2006) 179
  [arXiv:hep-ph/0507244].
  %%CITATION = HEP-PH 0507244;%%

%\cite{Bethke:2006ac}
\bibitem{Bethke:2006ac}
  S.~Bethke,
  %``Experimental tests of asymptotic freedom,''
  Prog.\ Part.\ Nucl.\ Phys.\  {\bf 58}, 351 (2007)
  [arXiv:hep-ex/0606035].
  %%CITATION = PPNPD,58,351;%%

%\cite{de Florian:2007hc}
\bibitem{de Florian:2007hc}
  D.~de Florian, R.~Sassot and M.~Stratmann,
  %``Global Analysis of Fragmentation Functions for Protons and Charged
  %Hadrons,''
  Phys.\ Rev.\  D {\bf 76}, 074033 (2007)
  [arXiv:0707.1506 [hep-ph]].
  %%CITATION = PHRVA,D76,074033;%%


\bibitem{Corcella:2005us}
  G.~Corcella and L.~Magnea,
  %``Soft-gluon resummation effects on parton distributions,''
  Phys.\ Rev.\  D {\bf 72} (2005) 074017
  [arXiv:hep-ph/0506278].

\bibitem{Martin:2007bv}
  A.~D.~Martin, W.~J.~Stirling, R.~S.~Thorne and G.~Watt,
  %``Update of Parton Distributions at NNLO,''
  Phys.\ Lett.\  B {\bf 652}, 292 (2007)
  [arXiv:0706.0459 [hep-ph]].

%\cite{Tung:2006tb}
\bibitem{Tung:2006tb}
  W.~K.~Tung, H.~L.~Lai, A.~Belyaev, J.~Pumplin, D.~Stump and C.~P.~Yuan,
  %``Heavy quark mass effects in deep inelastic scattering and global QCD
  %analysis,''
  JHEP {\bf 0702}, 053 (2007)
  [arXiv:hep-ph/0611254].
  %%CITATION = JHEPA,0702,053;%%

\bibitem{Martin:2004dh}
  A.~D.~Martin, R.~G.~Roberts, W.~J.~Stirling and R.~S.~Thorne,
  %``Parton distributions incorporating QED contributions,''
  Eur.\ Phys.\ J.\  C {\bf 39}, 155 (2005)
  [arXiv:hep-ph/0411040].

\bibitem{Ciafaloni:2000df}
  M.~Ciafaloni, P.~Ciafaloni and D.~Comelli,
  %``Bloch-Nordsieck violating electroweak corrections to inclusive TeV  scale
  %hard processes,''
  Phys.\ Rev.\ Lett.\  {\bf 84}, 4810 (2000)
  [arXiv:hep-ph/0001142].


\bibitem{Ciafaloni:2005fm}
  P.~Ciafaloni and D.~Comelli,
  %``Electroweak evolution equations,''
  JHEP {\bf 0511}, 022 (2005)
  [arXiv:hep-ph/0505047].
  %%CITATION = JHEPA,0511,022;%%


\bibitem{FortranPolyLog}
  T.~Gehrmann and E.~Remiddi,
  %``Numerical evaluation of two-dimensional harmonic polylogarithms,''
  Comput.\ Phys.\ Commun.\  {\bf 144} (2002) 200.
  %[hep-ph/0111255].
  %%CITATION = HEP-PH 0111255;%%

\bibitem{F95Explained}
  M. Metcalf and J. Reid, \emph{Fortran 90/95 Explained}, Oxford
  University Press, 1996.

\bibitem{F95WebResources} Many introductions and tutorials about
  fortran~90 may be found at
  \url{http://dmoz.org/Computers/Programming/Languages/Fortran/Tutorials/Fortran_90_and_95/}

%============= QED refs 
  
%\cite{Manohar:2016nzj}
\bibitem{Manohar:2016nzj}
A.~Manohar, P.~Nason, G.~P.~Salam and G.~Zanderighi,
%``How bright is the proton? A precise determination of the photon parton distribution function,''
Phys. Rev. Lett. \textbf{117} (2016) no.24, 242002
doi:10.1103/PhysRevLett.117.242002
[arXiv:1607.04266 [hep-ph]].

\bibitem{Manohar:2017eqh}
A.~V.~Manohar, P.~Nason, G.~P.~Salam and G.~Zanderighi,
%``The Photon Content of the Proton,''
JHEP \textbf{12} (2017), 046
doi:10.1007/JHEP12(2017)046
[arXiv:1708.01256 [hep-ph]].

%\cite{Buonocore:2020nai}
\bibitem{Buonocore:2020nai}
L.~Buonocore, P.~Nason, F.~Tramontano and G.~Zanderighi,
%``Leptons in the proton,''
JHEP \textbf{08} (2020) no.08, 019
doi:10.1007/JHEP08(2020)019
[arXiv:2005.06477 [hep-ph]].

%\cite{Buonocore:2021bsf}
\bibitem{Buonocore:2021bsf}
L.~Buonocore, P.~Nason, F.~Tramontano and G.~Zanderighi,
%``Photon and leptons induced processes at the LHC,''
JHEP \textbf{12} (2021), 073
doi:10.1007/JHEP12(2021)073
[arXiv:2109.10924 [hep-ph]].

%==============

%\cite{Roth:2004ti}
\bibitem{Roth:2004ti}
M.~Roth and S.~Weinzierl,
%``QED corrections to the evolution of parton distributions,''
Phys. Lett. B \textbf{590} (2004), 190-198
doi:10.1016/j.physletb.2004.04.009
[arXiv:hep-ph/0403200 [hep-ph]].


%\cite{Sborlini:2013jba}
\bibitem{Sborlini:2013jba}
G.~F.~R.~Sborlini, D.~de Florian and G.~Rodrigo,
%``Double collinear splitting amplitudes at next-to-leading order,''
JHEP \textbf{01} (2014), 018
doi:10.1007/JHEP01(2014)018
[arXiv:1310.6841 [hep-ph]].

\bibitem{Sborlini:2014mpa}
G.~F.~R.~Sborlini, D.~de Florian and G.~Rodrigo,
%``Triple collinear splitting functions at NLO for scattering processes with photons,''
JHEP \textbf{10} (2014), 161
doi:10.1007/JHEP10(2014)161
[arXiv:1408.4821 [hep-ph]].

\bibitem{Sborlini:2014kla}
G.~F.~R.~Sborlini, D.~de Florian and G.~Rodrigo,
%``Polarized triple-collinear splitting functions at NLO for processes with photons,''
JHEP \textbf{03} (2015), 021
doi:10.1007/JHEP03(2015)021
[arXiv:1409.6137 [hep-ph]].

%\cite{deFlorian:2015ujt}
\bibitem{deFlorian:2015ujt}
D.~de Florian, G.~F.~R.~Sborlini and G.~Rodrigo,
%``QED corrections to the Altarelli\textendash{}Parisi splitting functions,''
Eur. Phys. J. C \textbf{76} (2016) no.5, 282
doi:10.1140/epjc/s10052-016-4131-8
[arXiv:1512.00612 [hep-ph]].

%\cite{deFlorian:2016gvk}
\bibitem{deFlorian:2016gvk}
D.~de Florian, G.~F.~R.~Sborlini and G.~Rodrigo,
%``Two-loop QED corrections to the Altarelli-Parisi splitting functions,''
JHEP \textbf{10} (2016), 056
doi:10.1007/JHEP10(2016)056
[arXiv:1606.02887 [hep-ph]].
%51 citations counted in INSPIRE as of 23 Sep 2023

%\cite{ParticleDataGroup:2022pth}
\bibitem{ParticleDataGroup:2022pth}
R.~L.~Workman \textit{et al.} [Particle Data Group],
%``Review of Particle Physics,''
PTEP \textbf{2022} (2022), 083C01
doi:10.1093/ptep/ptac097

%\cite{Frixione:2023gmf}
\bibitem{Frixione:2023gmf}
S.~Frixione and G.~Stagnitto,
%``The muon parton distribution functions,''
[arXiv:2309.07516 [hep-ph]].

%\cite{Nason:1989zy}
\bibitem{Nason:1989zy}
P.~Nason, S.~Dawson and R.~K.~Ellis,
%``The One Particle Inclusive Differential Cross-Section for Heavy Quark Production in Hadronic Collisions,''
Nucl. Phys. B \textbf{327} (1989), 49-92
[erratum: Nucl. Phys. B \textbf{335} (1990), 260-260]
doi:10.1016/0550-3213(89)90286-1
%1222 citations counted in INSPIRE as of 25 Sep 2023


\bibitem{Cacciari:2015jma}
M.~Cacciari, F.~A.~Dreyer, A.~Karlberg, G.~P.~Salam and G.~Zanderighi,
%``Fully Differential Vector-Boson-Fusion Higgs Production at Next-to-Next-to-Leading Order,''
Phys. Rev. Lett. \textbf{115} (2015) no.8, 082002
[erratum: Phys. Rev. Lett. \textbf{120} (2018) no.13, 139901]
doi:10.1103/PhysRevLett.115.082002
[arXiv:1506.02660 [hep-ph]].

\bibitem{Dreyer:2016oyx}
F.~A.~Dreyer and A.~Karlberg,
%``Vector-Boson Fusion Higgs Production at Three Loops in QCD,''
Phys. Rev. Lett. \textbf{117} (2016) no.7, 072001
doi:10.1103/PhysRevLett.117.072001
[arXiv:1606.00840 [hep-ph]].

\bibitem{Dreyer:2018qbw}
F.~A.~Dreyer and A.~Karlberg,
%``Vector-Boson Fusion Higgs Pair Production at N$^3$LO,''
Phys. Rev. D \textbf{98} (2018) no.11, 114016
doi:10.1103/PhysRevD.98.114016
[arXiv:1811.07906 [hep-ph]].

\bibitem{Dreyer:2018rfu}
F.~A.~Dreyer and A.~Karlberg,
%``Fully differential Vector-Boson Fusion Higgs Pair Production at Next-to-Next-to-Leading Order,''
Phys. Rev. D \textbf{99} (2019) no.7, 074028
doi:10.1103/PhysRevD.99.074028
[arXiv:1811.07918 [hep-ph]].

\bibitem{bertonekarlberg}
  V. Bertone and A. Karlberg, to appear.
  
\bibitem{vanNeerven:1999ca}
W.~L.~van Neerven and A.~Vogt,
%``NNLO evolution of deep inelastic structure functions: The Nonsinglet case,''
Nucl. Phys. B \textbf{568} (2000), 263-286
doi:10.1016/S0550-3213(99)00668-9
[arXiv:hep-ph/9907472 [hep-ph]].

\bibitem{Vermaseren:2005qc}
J.~A.~M.~Vermaseren, A.~Vogt and S.~Moch,
%``The Third-order QCD corrections to deep-inelastic scattering by photon exchange,''
Nucl. Phys. B \textbf{724} (2005), 3-182
doi:10.1016/j.nuclphysb.2005.06.020
[arXiv:hep-ph/0504242 [hep-ph]].

\bibitem{vanNeerven:2000uj}
W.~L.~van Neerven and A.~Vogt,
%``NNLO evolution of deep inelastic structure functions: The Singlet case,''
Nucl. Phys. B \textbf{588} (2000), 345-373
doi:10.1016/S0550-3213(00)00480-6
[arXiv:hep-ph/0006154 [hep-ph]].

\bibitem{Davies:2016ruz}
J.~Davies, A.~Vogt, S.~Moch and J.~A.~M.~Vermaseren,
%``Non-singlet coefficient functions for charged-current deep-inelastic scattering to the third order in QCD,''
PoS \textbf{DIS2016} (2016), 059
doi:10.22323/1.265.0059
[arXiv:1606.08907 [hep-ph]].

\bibitem{Moch:2004xu}
S.~Moch, J.~A.~M.~Vermaseren and A.~Vogt,
%``The Longitudinal structure function at the third order,''
Phys. Lett. B \textbf{606} (2005), 123-129
doi:10.1016/j.physletb.2004.11.063
[arXiv:hep-ph/0411112 [hep-ph]].

\bibitem{Moch:2008fj}
S.~Moch, J.~A.~M.~Vermaseren and A.~Vogt,
%``Third-order QCD corrections to the charged-current structure function F(3),''
Nucl. Phys. B \textbf{813} (2009), 220-258
doi:10.1016/j.nuclphysb.2009.01.001
[arXiv:0812.4168 [hep-ph]].

%\cite{Moch:2021qrk}
\bibitem{Moch:2021qrk}
S.~Moch, B.~Ruijl, T.~Ueda, J.~A.~M.~Vermaseren and A.~Vogt,
%``Low moments of the four-loop splitting functions in QCD,''
Phys. Lett. B \textbf{825} (2022), 136853
doi:10.1016/j.physletb.2021.136853
[arXiv:2111.15561 [hep-ph]].
%32 citations counted in INSPIRE as of 05 Sep 2023

%\cite{Falcioni:2023luc}
\bibitem{Falcioni:2023luc}
G.~Falcioni, F.~Herzog, S.~Moch and A.~Vogt,
%``Four-loop splitting functions in QCD \textendash{} The quark-quark case,''
Phys. Lett. B \textbf{842} (2023), 137944
doi:10.1016/j.physletb.2023.137944
[arXiv:2302.07593 [hep-ph]].
%5 citations counted in INSPIRE as of 05 Sep 2023

%\cite{Falcioni:2023vqq}
\bibitem{Falcioni:2023vqq}
G.~Falcioni, F.~Herzog, S.~Moch and A.~Vogt,
%``Four-loop splitting functions in QCD -- The gluon-to-quark case,''
[arXiv:2307.04158 [hep-ph]].
%1 citations counted in INSPIRE as of 05 Sep 2023

%\cite{Gehrmann:2023cqm}
\bibitem{Gehrmann:2023cqm}
T.~Gehrmann, A.~von Manteuffel, V.~Sotnikov and T.~Z.~Yang,
%``Complete $N_f^2$ contributions to four-loop pure-singlet splitting functions,''
[arXiv:2308.07958 [hep-ph]].
%0 citations counted in INSPIRE as of 05 Sep 2023

%\cite{Blumlein:2022gpp}
\bibitem{Blumlein:2022gpp}
J.~Bl\"umlein, P.~Marquard, C.~Schneider and K.~Sch\"onwald,
%``The massless three-loop Wilson coefficients for the deep-inelastic structure functions F$_{2}$, F$_{L}$, xF$_{3}$ and g$_{1}$,''
JHEP \textbf{11} (2022), 156
doi:10.1007/JHEP11(2022)156
[arXiv:2208.14325 [hep-ph]].
%15 citations counted in INSPIRE as of 06 Sep 2023

%\cite{Falcioni:2023tzp}
\bibitem{Falcioni:2023tzp}
G.~Falcioni, F.~Herzog, S.~Moch, J.~Vermaseren and A.~Vogt,
%``The double fermionic contribution to the four-loop quark-to-gluon splitting function,''
Phys. Lett. B \textbf{848} (2024), 138351
doi:10.1016/j.physletb.2023.138351
[arXiv:2310.01245 [hep-ph]].
%2 citations counted in INSPIRE as of 04 Dec 2023

%\cite{Moch:2023tdj}
\bibitem{Moch:2023tdj}
S.~Moch, B.~Ruijl, T.~Ueda, J.~Vermaseren and A.~Vogt,
%``Additional moments and x-space approximations of four-loop splitting functions in QCD,''
[arXiv:2310.05744 [hep-ph]].
%3 citations counted in INSPIRE as of 04 Dec 2023

%\cite{Gehrmann:2023iah}
\bibitem{Gehrmann:2023iah}
T.~Gehrmann, A.~von Manteuffel, V.~Sotnikov and T.~Z.~Yang,
%``The $N_f \,C_F^3$ contribution to the non-singlet splitting function at four-loop order,''
[arXiv:2310.12240 [hep-ph]].
%0 citations counted in INSPIRE as of 04 Dec 2023

%\cite{Salam:2008qg}
\bibitem{Salam:2008qg}
G.~P.~Salam and J.~Rojo,
%``A Higher Order Perturbative Parton Evolution Toolkit (HOPPET),''
Comput. Phys. Commun. \textbf{180} (2009), 120-156
doi:10.1016/j.cpc.2008.08.010
[arXiv:0804.3755 [hep-ph]].
%271 citations counted in INSPIRE as of 04 Dec 2023

\end{thebibliography}


\end{document}

